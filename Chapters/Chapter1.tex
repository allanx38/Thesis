% Chapter 1

\chapter{Introduction} % Main chapter title

\label{Chapter1} % For referencing the chapter elsewhere, use \ref{Chapter1} 

\lhead{Chapter 1. \emph{Inroduction}} % This is for the header on each page - perhaps a shortened title

%----------------------------------------------------------------------------------------

\section{Background}
For hundreds of years speculators have tried to make a monetary profit in financial markets by predicting the future price of commodities, stocks, foreign exchange rates and more recently futures and options. Over the last few decades these efforts have increased markedly, using a variety of techniques \citep{Hsu201114026}, which can be broadly classified into three categories:

\begin{itemize}
\item fundamental analysis
\item technical analysis
\item traditional time series forecasting
\end{itemize}

\subsection{Fundamental Analysis}
Fundamental analysis makes use of basic market information in order to predict future movements of an asset. If an investor was looking at a particular stock's fundamental data they would consider information such as revenue, profit forecasts, supply, demand and operating margins etc. Speculators looking at commodities might consider weather patterns, political aspects, government legislation and so on. Effectively fundamental analysis is concerned with macro economic and political factors that might affect the future price of a financial asset. Fundamental analysis is not considered further in this study.

%%(\cite{Hsu201114026}) - \te (Thomsett, 1998). 

\subsection{Technical Analysis}
\label{chp1:ta}
Technical analysis is the study of historical prices and patterns with the aim of predicting future prices. Practitioners of technical analysis in the past were referred to as chartists, as they believed all that was needed to know about a particular market was contained in its pricing chart. \cite{murphy1999technical} defines technical analysis as:

%(\cite{Hsu201114026}) - \textquotedblleft technical analysis studies the stock prices and related issues,
%including analysis of recent and historical price trends, cycles
%and factors beyond the stock price, such as dividend payments,
%trading volume, index trends, industry group trends and popularity,
%and volatility of a stock (Thomsett, 1999). Technical analysis, rather than relying solely upon historical financial information,
%analysts will surmise upon recent trends in stock price changes,
%prices and earnings relationships, the activity volume of a particular
%stock or industry, and other similar indicators in order to determine
%changes in stocks, and in the market itself (Thomsett, 1999). \textquotedblright

\textit{\textquotedblleft Technical analysis is the study of market action, primarily through the use of charts for the purpose of forecasting future price trends.\textquotedblright}

Technical analysis (TA) is interesting as it tends to polarise opinion as to its scientific basis and effectiveness. To many people and particularly scholars in academia it is considered little more than Black Magic. Consider the words of \cite{malkiel1999random}:

\textit{\textquotedblleft Obviously I am biased against the chartist. This is not only a personal predilection, but a professional one as well. Technical Analysis is anathema to the academic world. We love to pick on it. Our bullying tactics are prompted by two considerations: (1) the method is patently false; and (2) it's easy to pick on. And while it may seem a bit unfair to pick on such a sorry target, just remember: it is your money we are trying to save.\textquotedblright}

However, in the world of finance technical analysis is ubiquitous and widely used \citep{Menkhoff20102573}. In support of TA a plethora of so-called indicators have been developed over the years from simple moving averages to much more exotic offerings. Today every piece of software or on-line analysis tool provides the ability to place a multitude of technical indicators on a graph of a stock, commodity or any financial instrument.

Most technical indicators essentially fall into one of two main categories, ones attempting to detect the start and direction of trends and those trying to identify market reversals generally called oscillators. Trend analysis indicators include Average Direction Index (ADX), Aroon, Moving Averages and Commodity Channel Indexes (CCI). Price oscillator indicators include, Moving Average Convergence Divergence (MACD - \citep{appel2007understanding}), Stochastics, Relative Strength Index (RSI) and the Chande Momentum Oscillator (CMO).

\subsection{Time Series Forecasting}
The study of forecasting time series data has been an active area of study for several decades \citep{DeGooijer2006443}. Series data is ordered such that the ordering is an important if not critical aspect of the data, with the requirement to maintain this ordering enforcing certain requirements on any processing. Series data can be ordered by factors such as distance or height but typically time is the ordering encountered. Financial data is an important category of series data and a variety of well known time series forecasting methods have been applied to the problem of predicting price movements in the financial markets. These have included, exponential smoothing, auto-regressive moving average (ARMA)
and auto-regressive integrated moving average (ARIMA).

A variety of smoothing algorithms have been applied to series data in general and financial data in particular. Moving averages, including simple, weighted and exponential, are widely employed by participants in financial markets to both predict future movements and quantify current conditions. Classical time series analysis such as so-called Holt-Winters exponential smoothing, the auto-regressive moving average (ARMA or Box-Jenkins model) and auto-regressive integrated moving average (ARIMA) methods have also been widely employed. In more recent years data mining techniques have been applied to the problem of financial time series prediction, for example with the use of artificial neural networks (ANNs) and support vector machines (SVM) as well as an hybrid approach of combining the classic time series techniques with the data mining methods in an attempt to leverage the strengths of each technique.

%(\cite{Hsu201114026}) - \textquotedblleft In addition, traditional time series forecasting techniques, such as
%autoregressive integrated moving average (ARIMA) (Box & Jenkins,
%1970), 
%generalized autoregressive conditional heteroskedasticity (GARCH) (Bollerslev, 1986), and 
% multivariate regression have been
%applied to the prediction of stock price movements. 

%In recent years,
%data mining/computational intelligence techniques have become
%another important approach to predict stock prices. For example,
%Kim and Han (2000) utilized genetic algorithms (GAs) to discretize
%features and determine the connection weights of artificial neural
%networks (ANNs), thus, predicting the stock price index. \textquotedblright


\section{Statement of the Problem}
The problem under study in this thesis is that of predicting the movement of financial markets. Financial markets include:
\begin{itemize}
\item Indices e.g.\ Dow Jones Index, FTSE 100 etc.
\item Commodities e.g.\ gold, oil etc.
\item Foreign exchange rates (also known as Forex or FX) e.g.\ GBP\\USD (price of British pounds divided by US dollars).
\item Stocks e.g.\ Google, Apple, Barclays Bank etc.
\end{itemize}
The goal of financial traders is to detect the movement of the markets and buy instruments expected to rise in price \textquotedblleft going long" and sell those predicted to fall in price \textquotedblleft going short". The markets are a neutral sum process, for every participant who gains there are those who lose.

\section{Purpose of Study}
The purpose of this study is to investigate and establish the usefulness and accuracy of a selection of technical indicators and time series analysis on the ability to predict future data movements in a group of national indice data sets.

\subsection{Study Objectives}
The objective of this study is two fold:

\begin{enumerate}
\item To determine if a group of popular and widely used technical indicators can be used to predict the direction of movement in a range of financial markets.
\item To investigate if traditional time series models can predict the direction of movement in a range of financial markets.
\end{enumerate}

%Arima vs Holt Winters - ID trend - trad tools

%Are there any other - Neural Nets

\section{Research Question}
The research question addressed in this study is: 

\textit{\textquotedblleft Can the use of technical indicators or time series analysis help to predict the future direction and movement of financial markets?\textquotedblright}

\section{Methodology}
The following methodology was used to answer the research question:

\begin{enumerate}
\item The current research in the field was reviewed.
\item Appropriate data was collected, primarily from freely available sources on the internet such as Yahoo and Google.
\item Initial data investigations and visualisations were carried out on the data.
\item Based on initial analysis, \textquotedblleft base line" systems were established that could be used to compare the performance of systems generated by technical analysis and time series modelling.
\item A number of trading algorithms were generated that consumed the output of technical analysis indicators to determine in which direction to trade the financial data at any one time.
\item Times series modelling methods were used to generate forecasts for the financial data, and these were used in trading algorithms to decide in which direction, long or short, to trade.
\end{enumerate}

In summary, the output of technical and times series analysis was consumed in a range trading algorithms. The decisions regarding which direction to trade a particular stock market was based on the predictions or output of the analysis. Success was measured in terms of whether the trading systems developed can profitably predict which way to trade the financial markets.

\section{Limitations of the Study}
%Are there factors outside your control, technical or non-technical, which may limit this study? 
Limitations in this study include:
\begin{enumerate}
\item Choice of technical indicators - a small selection of the huge number available was selected. The selected group represent widely used examples and are drawn from the various categories available.
\item Use of financial data relating to stock market indices - daily data in the form of open, high, low and close prices (OHLC) from national stock market indices such as the Dow Jones or FTSE 100 is readily and freely available and thus was used in this study. High quality data in time frames other than daily or from alternative financial markets such commodities or foreign exchange is generally only commercially available and beyond the resources of this study.
\end{enumerate}

\section{Scope of the Study }
%Are the aspects of your problem statement that cannot be addressed in the time available? Clarify what will / will not be addressed by this project.
There is a huge choice of financial data sets from which to choose and likewise many dozens of technical indicators. Given the time frame and resources available, this study employed daily data from major national indices such as the German DAX, US Dow and Japanese Nikkei. Technical indicators selected included examples from the primary categories such as trend detection and market reversal indicators.

\section{Structure of Project}
Chapter \ref{Chapter2} is a literature review and introduction to financial market trading, the methods and theory of technical analysis and time series modelling. Financial trading systems in general are discussed along with the use and applicability of technical analysis. The classical time series methods of Holt-Winters exponential smoothing, auto-regressive moving average (ARMA or Box-Jenkins model) and auto-regressive integrated moving average (ARIMA) are introduced and explained along with more recent developments such as hybrid ARIMA models.

Chapter \ref{Chapter3} introduces the methodology used in this study. It includes a description of the data sets employed, software and programming languages utilised and the general approach taken. Chapter \ref{Chapter4} details the experiments carried out using a variety of technical analysis indicators and lists the results from the trading algorithms generated. Chapter \ref{Chapter5} documents the experimental work based on the use of time series modelling to generate forecasts for the financial data sets.

Chapter \ref{Chapter6} is an analysis of the results obtained in Chapters \ref{Chapter4} and \ref{Chapter5} along with conclusions and suggestions for future work. Appendix \ref{AppendixA} lists all the R programming code used in study. Wherever possible this report has the analysis generated by R programming code embedded into it. Thus, all trading algorithms coded in R detailed in Chapters \ref{Chapter4} and \ref{Chapter5} generate results that are dynamically embedded into this report. An update or alteration of this code followed by recompilation of this manuscript updates the tables and results accordingly. 

Appendix \ref{AppendixB} provides additional details and background information on various technical analysis indicators. Finally, Appendix \ref{AppendixC} presents all the results generated in Chapters \ref{Chapter4} and \ref{Chapter5} collected together by the particular financial market. 
