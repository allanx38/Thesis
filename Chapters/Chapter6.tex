% Chapter 6

\chapter{Analysis} % Main chapter title

\label{Chapter6} % For referencing the chapter elsewhere, use \ref{Chapter6} 

\lhead{Chapter 6. \emph{Analysis}} % This is for the header on each page - perhaps a shortened title

%----------------------------------------------------------------------------------------

\section{Introduction}
In chapters 4 and 5 a wide variety of analytical techniques were applied to a variety of time series data sets. In Chapter 4 a range of trading algorithms were developed based on technical analysis indicators.  The intention was to automate the decision of whether to buy or sell a market based on the value of the indicator. For comparison purposes, two simple so called "naive" systems were explored to set a base line against which the technical analysis indicators could be compared.  The technical indicators were grouped together in their general area of applicability, namely trend detection indicators, reversal, momentum and candlestick indicators.

Chapter 5 continued the exploration of financial time series through the use of exponential smoothing, ARIMA and hybrid ARIMA techniques. The generated models were used to create one-step forecasts which were then combined with the original data set. These data sets were then fed into trading algorithms which used the forecast values to make trading decisions. Again a series of simple forecast techniques were used as a baseline against which the trading system algorithms could be compared.

\section{Technical Analysis}
Initially two simple, naive systems were explored to set a baseline for further analysis. These systems were the Naive Long System which mirrors a buy and hold strategy and a Naive Follow Previous system which simply repeats the previous days market direction.

\subsection{Naive Systems}
The first base line system tried was the Naive Long system in which a market buy is placed each day and is similar to the so-called "Buy and Hold" technique. The assumption here is that the market rises over time and if an investor simply holds a security it will eventually generate a profit.  The total profit is simply the price at the start, in this case the data set started in 2000, subtracted from the price at the end of the period which in this case was the end of 2013.

The first iteration of the algorithm placed a buy at the start of the trading session and closed it at the end and thus the system was out of the market overnight. This resulted in significant discrepancies from the returns expected from a buy and hold system. With a buy and hold system the returns would have been as follows:

\begin{itemize}
\item Dax: +2591
\item CAC:-1774
\item FTSE: -181
\item Dow: +11501
\item Nikkei: -2649
\item AORD: +2201
\end{itemize}

From simply trading long during market hours the Dax generated a loss as opposed to the 2591 profit expected, likewise the CAC showed a much larger loss than expected and the Nikkei resulted in a large loss when a small loss was expected. The Dow, FTSE and AORD were similar to the expected values. Changing the algorithm such that the trades ran close to close and covered the full 24 hour period resulted in system results that matched the expected results from a buy and hold approach.  Clearly the discrepancies from the first algorithm were due to the relative amounts the markets moved during the day as opposed to during the \textquotedblleft out of hours" trading. There is a slight bias for the markets to move upwards overnight and over the course of the study (14 years) adds up to significant values.  

TO DO - Mkt naming consistency: FTSE, AORD, Nikkei 
TO DO - PL per trade - L / S

The second naive system was termed "Naive Follow Previous" and simply places a trade today consistent with the market direction from the previous day. This idea produced very poor results, with every market losing money. Clearly if the trades were reversed so that the algorithm traded in the opposite direction to the previous day the exact opposite results would have occurred (in a real scenario this wouldn't be true because of trading charges).  From these results it can be concluded that the markets have a tendency to "flip flop" and reverse back on themselves, and the phenomena of market reverses is well understood.

Naive Reversed PL:
CAC - 7800 (L)
FTSE - 4000 (LS)
Dow - 6000/15800
Nik - 20500(L)

\subsection{Trend Detection}
The first group of the technical analysis indicators studied were the trend detection indicators. identification of trend direction and strength is very important in the world of financial trading and one of the most widely encountered phrases is "the trend is your friend" as most authorities advocate trading in the direction of the trend. (In fact on a recent webinar it was claimed that 80\% of all money made is made trading in the direction of the trend.)  Well known indicators that purport to assist the trader in identifying trends are the simple moving average (SMA), the moving average convergence/divergence indicator (MACD) and the Aroon indicator.

The use of simple moving average is wide-spread in the financial markets. Market participants track moving averages or even more than one and make a decision which way to trade based on the position of the current price relative to it. Popular values to use in the SMA are 25, 50 and 200. The results of a trading based on SMA is presented in table \ref{}. The algorithm places a buy trade if the current price is above the SMA and a sell trade if it is below it. The results from this algorithm an be seen in 

Results are mixed with some markets producing positive results when trades are directed by the SMA values and some result in losses. The German Dax produces positive results across all the SMA values with values from trading short (predictng the market will decline doing best).  The French CAC displays different results all the SMA values produce negative results in trying to  predict long trades but positive results when trying to predict short results.  The UK's FTSE 100 displays different behaviour again, producing negative results accross the board. The Dow produces a different set of results again, trades on the long side prodce a profit whereas trading short results in losses.  The Japanese Nikkei exhibits similar results to the CAC in that short trades are profiatable whereas long trades aren't and finally the Australian AORD is similar to the DAX in producing positive results across the board. DOES this reflect on how these markets move? - smooth trends would be profitable ...

NOTE - worth analysing "trending" behaviour?  Amount of time >70 in aroon?   

27/05
General – how about a “random” system, random buys and sells, and random with SLoss …


SMA cont:
The results from this algorithm can be seen in Table ref{} and are mixed

b. applied SL – winners and losers, need to assess ….  Do we want the SLoss discussion moved here?

b. MACD – no SL applied here
MACD can generally be used 2 ways, as a trend detection indicator and as an over-bought / over-sold indicator in which case traders use it to identify potential market reversals. In this section the indicator was used as a trend detector and the results from a system based on the MACD indicator can be seen in Table \ref{}.  The algorithm trades long when the value of MACD is greater than the value of the signal line, see section \ref{} for more details of the implementation of the MACD indicator. (Is it in the App?). The results are not very impressive, only the Nikkei producing reasonable profits – though wouldn’t beat the flip flop naive system.

c. Aroon – alone and SL -> sort ref out ...
The final trend detection indicator examined was Aroon.  This indicator measures the time since the previous high or low within a certain time window. The algorithms presented here used a time window of 20. If today was highest price in the last 20 days trading the indicator would take a value of 100 and for each day that proceeds that doesn't make a new high the indicator falls by 5 (100 divided by the lag period which is 20).  Thus if the highest price was four days ago the AroonUp value would be 80. The opposite situation occurs with regard to the low price. A value of 70 or above for the AroonUp is indicative of a upward trending market and likewise a value of 70 and above for AroonDn suggests a falling market.

Naive Reversed PL:
CAC - 7800 (L)  -> 
FTSE - 4000 (LS)
Dow - 6000/15800
Nik - 20500(L)

The results from an algorithm using these concepts can be seen in Table \ref{}. Overall the results are encouraging.  The Dax, FTSE, Dow, AORD produce positive returns for both Long and Short. The CAC and Nikkei are positive short trading. 

The use of a Stop Loss improves the returns from all markets except the Dow. One again using a SL with the Dow shows very marked negative impacts on profits. These results can be seen in Table \ref{}

MACD – no SL applied here
The results from a system based on the moving average convergence/divergence (MACD) indicator can be seen in Table \ref{}. The results are not very impressive, only the Nikkei producing reasonable profits – though wouldn’t beat the flip flop system.

Aroon – alone and SL
Alone – promising results on short side, long side CAC and Nik show loss
With SL – results improve, except Dow -> results get worse!!  See diff table.

\subsubsection{Market Reversal Indicators} – no SL used
In this section  two indicators that purport to assist in identifying market reversals are examined.

1. Parabolic Stop-and-Reverse (SAR)
The first market reversal indicator used was the Parabolic Stop-and-Reverse (SAR) an indicator initially developed for traders who were always in the market either long or short. The SAR was used to judge when the  position should be reversed from long to short or vice versa. The trading algorithm using the SAR trades each day (i.e opens a trade at the start of the trading session and closes it out at the end) and makes a decision regarding the direction of the trade based on the SAR indicator. If the market opening is above the SAR a long trade is initiated and vice versa if the market is below the SAR value.

The results from the trading system based on the SAR can be seen in Table \ref{} and are very poor. Only the Nikkei trading short produce reasonable results, but these are much worse than the naive flip flop method of section \ref{}.

2.	MACD
As previously mentioned the MACD indicator can be used as a market reversal indicator. Once the MACD value reaches its extreme values the market is considered over-bought or over-sold. The trading algorithm using this concept expects a market reversal once the MACD crosses above the 85\% quantile (of the MACD range) or the 15\% quantile. Short trades are initiated once the MACD crosses above the 85\% quantile value and short trades once it has passed below the 15\% quantile.

The results from this trading system can be seen in Table \ref{} and are very unimpressive. 

V modest results, but we are deffo not in trend at this point …  Cross-ref to a trend indicator?

\subsection{Momentum Indicators}

A third type of technical indicators are the momentum indicators, which are related to the trend detection indicators. Two such indicators are studied here, the stochastic and Rate of Change (ROC). 
1.	Stoch – plus SL
The stochastic oscillator is one of the oldest and most widely used of the techical indicators. Essentially it measures the percentage position the current close in relation to the high low range of period of interest. Thus it has conceptual similarities to the Aroon indicator. The stochastic is usually represented by two lines \%K which is the position of the price within this high low envelope described above, and \%D a moving average of \%K (see Chapter 4.  Appendix \ref{} for more details). Long trades are initiated when \%K is above \%D and vice versa for short trades. Results from an algorithm implementing these ideas can be seen in Table \ref{}. the results are poor being significantly worse than the naive flip flop system. 

NB – written “of interest” twice … also "This is of intrest as ... " , "wihtin this high"

Use of the SL improves things - except the Dow as usual.

1.	Stoch – plus SL
NB – written “of interest” twice …
a.	Alone – results v. modest


2.	ROC
-	Needs finishing in Chp 4
-	Roc1 and 2 – what’s diff ? let's go with ROC2 ...

ROC1 - Mkt$Long <- ifelse(Mkt$prevROC < lw,Mkt$Close-Mkt$Open,NA)

ROC2 - Mkt$Long <- ifelse(Mkt$prevROC > 0,Mkt$Close-Mkt$Open,NA)

The second momentum indicators is the Rate Of Change (ROC) indicator, and this is simply the difference between the current price and a price a certain number of days previously. If this value is positive the market is considered to be trending up and the larger the value the greater the trending momentum. The results from an algorithm using these ideas is presented in Table \ref{}. Most of the results are negative and in fact doing the opposite would produce reasonable results.


\subsection{Break-out systems}

The fourth section explored the idea of trade signals being generated by a particular value from the previous day, so-called brekout systems. Two particular values are used as the trigger price for a trade, the previous day's high/low or the 90\% quantile of the minor move (see section x). 

%1.	Daily high -> no SL …
%The results from using the previous day's high price or low price as a trigger to trade long or short can be seen in table \ref{]. Generally the results are very good with the exception of the Dow. These results can be linked to the data exploratory work shown in section \ref{} Table \ref{}. The best pl was shown in the Nikkei and this market had the highest number of times closing outside the previous high / low. Conversely, the lowest ranked market was the Dow and this produced negative results in the high/low break-out system.

%-	Error “seen in See Appendix A section A.1.4.1”, "andthis tallies "
%-	Good results – except Dow, reference to Chp3

2.	90% Q system – no SL
The second break-out sytem used the minor fluctuation 90\% quantile value as the level to buy or sell. Once the market moved this amount above
-	Good results, not as good as b/out system … Dow poor again … Oz v. good given price of mkt, almost as good as b/out sys 

2.	90% Q system – no SL
-	Good results, not as good as b/out system … Dow poor again … Oz v. good given price of mkt, almost as good as b/out sys …

\subsection{Candlestick Patterns}
1.	Hammer
2.	Hammer plus aroon

3.	Engulfing
4.	Engulfing plus aroon
To Do – name of tables are the same …

5.	Doji


Summary
SL – sma, aroon and stoch – why these?  Add to b/out and naïve systems



\section{Time Series Analysis}
Predicting either the closing price or whether the market will rise or fall, in which there are 2 options up/down represented as categorical U of D or represented as 1 and 0.

\subsection{Exponential moving Average - Predicting Closing Price}


\subsection{ARIMA - Predicting Closing Price}


\subsection{ARIMA hybrids - Predicting Closing Price}

\subsubsection{ARIMA / ANN}

System 1 - not very good overall, Dow long, Nik, CAC short ok.

System 2 - much worse than system 1 - CAC, Nik Long and Dow short - the inverse would be good candidates ...

\subsubsection{ARIMA / knn}

System 1 - good results - across the board.  TO DO - Stop Loss ...

system 2 - mixed.  Dax good, CAC, Nik short, Dow long OK.

\subsection{ARIMA hybrids - Predicting Up Down - Categorical}

\subsubsection{ARIMA / ANN}

\subsubsection{ARIMA / knn}

\subsubsection{ARIMA / SVN}

\subsection{ARIMA hybrids - Predicting Up Down - 01}

\subsubsection{ARIMA / ANN}

\subsubsection{ARIMA / ANN}

\subsubsection{ARIMA / knn}

\subsubsection{ARIMA / SVN}

\section{Conclusion}


\section{Future Work}

candlestick systems -> price 2,,3,4 days ahead?
