% Chapter 6

\chapter{Analysis} % Main chapter title

\label{Chapter6} % For referencing the chapter elsewhere, use \ref{Chapter6} 

\lhead{Chapter 6. \emph{Analysis}} % This is for the header on each page - perhaps a shortened title

%----------------------------------------------------------------------------------------

\section{Introduction}
In chapters 4 and 5 a wide variety of analytical techniques were applied to a variety of time series data sets. In Chapter 4 a range of trading algorithms were developed based on technical analysis indicators.  The intention was to automate the decision of whether to buy or sell a market based on the value of the indicator. For comparison purposes, two simple so called "naive" systems were explored to set a base line against which the technical analysis indicators could be compared.  The technical indicators were grouped together in their general area of applicability, namely trend detection indicators, reversal, momentum and candlestick indicators.

Chapter 5 continued the exploration of financial time series through the use of exponential smoothing, ARIMA and hybrid ARIMA techniques. The generated models were used to create one-step forecasts which were then combined with the original data set. These data sets were then fed into trading algorithms which used the forecast values to make trading decisions. Again a series of simple forecast techniques were used as a baseline against which the trading system algorithms could be compared.

\section{Technical Analysis}
Initially two simple, naive systems were explored to set a baseline for further analysis. These systems were the Naive Long System which mirrors a buy and hold strategy and a Naive Follow Previous system which simply repeats the previous days market direction.

\subsection{Naive Systems}
The first base line system tried was the Naive Long system in which a market buy is placed each day and is similar to the so-called "Buy and Hold" technique. The assumption here is that the market rises over time and if an investor simply holds a security it will eventually generate a profit.  The total profit is simply the price at the start, in this case the data set started in 2000, subtracted from the price at the end of the period which in this case was the end of 2013.

The first iteration of the algorithm placed a buy at the start of the trading session and closed it at the end and thus the system was out of the market overnight. This resulted in significant discrepancies from the returns expected from a buy and hold system. With a buy and hold system the returns would have been as follows:

\begin{itemize}
\item Dax: +2591
\item CAC:-1774
\item FTSE: -181
\item Dow: +11501
\item Nikkei: -2649
\item AORD: +2201
\end{itemize}

From simply trading long during market hours the Dax generated a loss as opposed to the 2591 profit expected, likewise the CAC showed a much larger loss than expected and the Nikkei resulted in a large loss when a small loss was expected. The Dow, FTSE and AORD were similar to the expected values. Changing the algorithm such that the trades ran close to close and covered the full 24 hour period resulted in system results that matched the expected results from a buy and hold approach.  Clearly the discrepancies from the first algorithm were due to the relative amounts the markets moved during the day as opposed to during the \textquotedblleft out of hours" trading. There is a slight bias for the markets to move upwards overnight and over the course of the study (14 years) adds up to significant values.  

The second naive system was termed "Naive Follow Previous" and simply places a trade today consistent with the market direction from the previous day. This idea produced very poor results, with every market losing money. Clearly if the trades were reversed so that the algorithm traded in the opposite direction to the previous day the exact opposite results would have occurred (in a real scenario this wouldn't be true because of trading charges).  From these results it can be concluded that the markets have a tendency to "flip flop" and reverse back on themselves, and the phenomena of market reverses is well understood.

%Naive Reversed PL:
%CAC - 7800 (L)
%FTSE - 4000 (LS)
%Dow - 6000/15800
%Nik - 20500(L)

\subsection{Trend Detection}
The first group of the technical analysis indicators studied were the trend detection indicators. Identification of trend direction and strength is very important in the world of financial trading and one of the most widely encountered phrases is "the trend is your friend", as most authorities advocate trading in the direction of the trend. (In fact on a recent webinar it was claimed that 80\% of all money made is made trading in the direction of the trend.)  Well known indicators that purport to assist the trader in identifying trends are the simple moving average (SMA), the moving average convergence/divergence indicator (MACD) and the Aroon indicator.

The use of simple moving average is wide-spread in the financial markets. Market participants track moving averages or even more than one and make a decision which way to trade based on the position of the current price relative to it. Popular values to use in the SMA are 25, 50 and 200. The results of a trading based on SMA is presented in Table \ref{tab:sma_results} of Chapter \ref{Chapter4}. The algorithm places a buy trade if the current price is above the SMA and a sell trade if it is below it. 

Results are mixed from using the SMA, with some markets producing positive results and some ending in losses. The German Dax produces positive results across all the SMA values with values from trading short (predicting the market will decline doing best).  The French CAC displays different results in that all the SMA values produce negative results in trying to  predict long trades but positive results when trying to predict short results.  The UK's FTSE 100 displays different behaviour again, producing negative results across the board. The Dow produces a different set of results again, trades on the long side produce a profit whereas trading short results in losses.  The Japanese Nikkei exhibits similar results to the CAC in that short trades are profitable whereas long trades aren't and finally the Australian AORD is similar to the DAX in producing positive results across the board. 

%DOES this reflect on how these markets move? - smooth trends would be profitable ...
%NOTE - worth analysing "trending" behaviour?  Amount of time >70 in aroon?   

In an attempt to improve the returns from the trading system a stop loss was introduced. Comparing Tables \ref{tab:sma_results} and \ref{tab:sma_results_Sloss} of Chapter \ref{Chapter4}
it can be seen that applying the stop loss has been on the whole beneficial to the results obtained, with the exception of those from the Dow which were negatively impacted. Essentially losing trades have been truncated while winning trades have been left to develop. This general pattern of a stop loss being beneficial to all the markets except the US Dow was seen multiple times with the systems tested.

%b. MACD – no SL applied here
The second trend detection indicator explored was the Moving Average Convergence Divergence (MACD) indicator, full details of which can be found in section \ref{appB:MACD} of Appendix \label{AppendixB}. The MACD can generally be used two ways, as a trend detection indicator and as an over-bought/over-sold indicator in which case traders use it to identify potential market reversals. In this section the indicator was used as a trend detector and the results from a system based on the MACD indicator can be seen in Table \ref{tab:mac_trend_results} in Chapter \ref{Chapter4}.  The algorithm trades long when the value of MACD is greater than the value of the signal line, see Appendix \ref{AppendixA} section \ref{appA:macd_xo} for details of the R code used. The results are not very impressive, only the Nikkei producing reasonable profits, although wouldn't beat the baseline Naive Reversing system.

c. Aroon – alone and SL -> sort ref out ...
The final trend detection indicator examined was Aroon.  This indicator measures the time since the previous high or low within a certain time window. The algorithms presented here used a time window of 20. If today was highest price in the last 20 days trading the indicator would take a value of 100 and for each day that proceeds that doesn't make a new high the indicator falls by 5 (100 divided by the lag period which is 20).  Thus if the highest price was four days ago the AroonUp value would be 80. The opposite situation occurs with regard to the low price. A value of 70 or above for the AroonUp is indicative of a upward trending market and likewise a value of 70 and above for AroonDn suggests a falling market.

Naive Reversed PL:
CAC - 7800 (L)  -> 
FTSE - 4000 (LS)
Dow - 6000/15800
Nik - 20500(L)

The results from an algorithm using these concepts can be seen in Table \ref{}. Overall the results are encouraging.  The Dax, FTSE, Dow, AORD produce positive returns for both Long and Short. The CAC and Nikkei are positive short trading. 

The use of a Stop Loss improves the returns from all markets except the Dow. One again using a SL with the Dow shows very marked negative impacts on profits. These results can be seen in Table \ref{}

MACD – no SL applied here
The results from a system based on the moving average convergence/divergence (MACD) indicator can be seen in Table \ref{}. The results are not very impressive, only the Nikkei producing reasonable profits – though wouldn't beat the flip flop system.

Aroon – alone and SL
Alone – promising results on short side, long side CAC and Nik show loss
With SL – results improve, except Dow -> results get worse!!  See diff table.

\subsection{Market Reversal Indicators} – no SL used
In this section  two indicators that purport to assist in identifying market reversals are examined.

1. Parabolic Stop-and-Reverse (SAR)
The first market reversal indicator used was the Parabolic Stop-and-Reverse (SAR) an indicator initially developed for traders who were always in the market either long or short. The SAR was used to judge when the  position should be reversed from long to short or vice versa. The trading algorithm using the SAR trades each day (i.e opens a trade at the start of the trading session and closes it out at the end) and makes a decision regarding the direction of the trade based on the SAR indicator. If the market opening is above the SAR a long trade is initiated and vice versa if the market is below the SAR value.

The results from the trading system based on the SAR can be seen in Table \ref{} and are very poor. Only the Nikkei trading short produce reasonable results, but these are much worse than the naive flip flop method of section \ref{}.

2.	MACD
As previously mentioned the MACD indicator can be used as a market reversal indicator. Once the MACD value reaches its extreme values the market is considered over-bought or over-sold. The trading algorithm using this concept expects a market reversal once the MACD crosses above the 85\% quantile (of the MACD range) or the 15\% quantile. Short trades are initiated once the MACD crosses above the 85\% quantile value and short trades once it has passed below the 15\% quantile.

The results from this trading system can be seen in Table \ref{} and are very unimpressive. 

V modest results, but we are deffo not in trend at this point …  Cross-ref to a trend indicator?

\subsection{Momentum Indicators}

A third type of technical indicators are the momentum indicators, which are related to the trend detection indicators. Two such indicators are studied here, the stochastic and Rate of Change (ROC). 
1.	Stoch – plus SL
The stochastic oscillator is one of the oldest and most widely used of the technical indicators. Essentially it measures the percentage position the current close in relation to the high low range of period of interest. Thus it has conceptual similarities to the Aroon indicator. The stochastic is usually represented by two lines \%K which is the position of the price within this high low envelope described above, and \%D a moving average of \%K (see Chapter 4.  Appendix \ref{} for more details). Long trades are initiated when \%K is above \%D and vice versa for short trades. Results from an algorithm implementing these ideas can be seen in Table \ref{}. the results are poor being significantly worse than the naive flip flop system. 

NB – written “of interest” twice … also "This is of interest as ... " , within this high"

Use of the SL improves things - except the Dow as usual.

1.	Stoch – plus SL
NB – written “of interest” twice …
a.	Alone – results v. modest


2.	ROC
-	Needs finishing in Chp 4
-	Roc1 and 2 – what's Diff ? let's go with ROC2 ...

ROC1 - Mkt$Long <- ifelse(Mkt$prevROC < lw,Mkt$Close-Mkt$Open,NA)

ROC2 - Mkt$Long <- ifelse(Mkt$prevROC > 0,Mkt$Close-Mkt$Open,NA)

The second momentum indicators is the Rate Of Change (ROC) indicator, and this is simply the difference between the current price and a price a certain number of days previously. If this value is positive the market is considered to be trending up and the larger the value the greater the trending momentum. The results from an algorithm using these ideas is presented in Table \ref{}. Most of the results are negative and in fact doing the opposite would produce reasonable results.


\subsection{Break-out systems}

The fourth section explored the idea of trade signals being generated by a particular value from the previous day, so-called breakout systems. Two particular values are used as the trigger price for a trade, the previous day's high/low or the 90\% quantile of the minor move (see section x). 

%1.	Daily high -> no SL …
%The results from using the previous day's high price or low price as a trigger to trade long or short can be seen in table \ref{]. Generally the results are very good with the exception of the Dow. These results can be linked to the data exploratory work shown in section \ref{} Table \ref{}. The best pl was shown in the Nikkei and this market had the highest number of times closing outside the previous high / low. Conversely, the lowest ranked market was the Dow and this produced negative results in the high/low break-out system.

%-	Error “seen in See Appendix A section A.1.4.1”, "andthis tallies "
%-	Good results – except Dow, reference to Chp3

2.	90% Q system – no SL
The second break-out stem used the minor fluctuation 90\% quantile value as the level to buy or sell. Once the market moved this amount above
-	Good results, not as good as b/out system … Dow poor again … Oz v. good given price of mkt, almost as good as b/out sys 

2.	90% Q system – no SL
-	Good results, not as good as b/out system … Dow poor again … Oz v. good given price of mkt, almost as good as b/out sys …

\subsection{Candlestick Patterns}
1.	Hammer
2.	Hammer plus aroon

3.	Engulfing
4.	Engulfing plus aroon
To Do – name of tables are the same …

5.	Doji


Summary
SL – sma, aroon and stoch – why these?  Add to b/out and naïve systems



\section{Time Series Analysis}
ARIMA and hybrid ARIMA models were used to generate forecasts of the closing prices and the more general situation of whether the market would rise or fall. In modelling the more general situation of market direction a categorical and a continuous label was employed. The categorical label used \textquotedblleft U" to represented occasions when the market prices increased and \textquotedblleft D" for when it decreased in value. Alternatively the values 1 and 0 were also used to represent up and down respectively. The primary difference between the two labels was in the values returned from the hybrid ARIMA models. When using 1 and 0 for the class label the models return a value in the range of 1 to 0, whereas for the categorical value there was only the choice of the two values. 

\subsection{Automatically generated ARIMA Models}
The auto.arima() function of the R forecast package was used to generate ARIMA models for the national indice data sets used in this study. For convenience the models picked are listed in Table \ref{tab:chp_ts_arima_models}.

%label - tab:chp_ts_arima_models
% latex table generated in R 3.1.0 by xtable 1.7-3 package
% Tue May 27 13:22:44 2014
\begin{table}[ht]
\centering
\caption[Arima models.]{Arima models  from national indices.} 
\label{tab:chp_ts_arima_models}
\begin{tabular}{lc}
  \toprule Market & Arima Model \\ 
  \midrule Dax & ARIMA(3,1,3)                    \\ 
  CAC & ARIMA(2,1,3)                    \\ 
  F100 & ARIMA(2,1,3)                    \\ 
  Dow & ARIMA(1,1,2)                    \\ 
  Nik & ARIMA(2,1,3)                    \\ 
  Oz & ARIMA(1,1,0)                    \\ 
   \bottomrule \end{tabular}
\end{table}


The one-step forecasts generated from these models were then used in two trading systems. In the first algorithm the decision to trade long or short was dependant upon on the relative values of the previous close price and the forecast. If the forecast was was higher than the close price a long trade wa entered in the expectation that the market would rise towards the prediction. The opposite situation was expected for when the forecast was lower than the close price. The second trading algorithm used the relative values of the predictions themselves in order to decide whether to trade long or short. If the current forecast was higher than the previous one a long trade was made and vice versa.

The results from both systems were poor. The difference in mean PL per trade between the first system based on the auto.arima models (previous close in comparison to forecast) and the mean PL for the Naive Reversing system from section \ref{sec:naive:rev} Chapter \ref{Chapter4} can be seen in Table \ref{tab:chp_ts:arima1_diff}. Most of the results are worse than the naive baseline system except for the French CAC and US Dow when trading long.

%label - tab:chp_ts:arima1_diff
% latex table generated in R 3.1.0 by xtable 1.7-3 package
% Sat Aug 16 11:48:05 2014
\begin{table}[ht]
\centering
\caption[Mean PL from ARIMA models minus mean PL from Naive Reverse system]{Mean Long/Short PL from system using predictions from ARIMA models with the results from the Naive Reverse system subtracted.} 
\label{tab:chp_ts:arima1_diff}
\begin{tabular}{lcc}
  \toprule Mkt & Diff in Mean Long PL & Diff in Mean Short PL \\ 
  \midrule DAX & -6 & -11 \\ 
  CAC & 2 & -3 \\ 
  FTSE & 1 & -3 \\ 
  Dow & -3 & -14 \\ 
  Nikkei & 20 & 1 \\ 
  AORD & 1 & -1 \\ 
   \bottomrule \end{tabular}
\end{table}


\subsection{ARIMA Hybrids - Predicting Closing Price}
Hybrid ARIMA models in which Artificial Neural Networks and k-Nearest Neighbour algorithms were used instead of regression in the ARIMA algorithm were used to predict the closing prices of financial markets. 

\subsubsection{ARIMA/Artificial Neural Networks (ANN)}
Overall the use of the models generated from hybrid ARIMA/ANN algorithms to create trading systems was not very successful. The results from passing the indice data sets augmented with a forecast attribute generated by the hybrid ARIMA models can be seen in Tables  \ref{tab:chp_ts:arima_ann_sys1} and \ref{tab:chp_ts:arima_ann_sys2} of Chapter \ref{Chapter5}. System 1 compares the price of the forecast with the price of the previous and in the event that the prediction is higher than the previous closing price a long trade is entered. The opposite is true when the forecast is lower than the closing price and a short trade is made. System 2 is similar but compares the forecast with the last forecast. In the event that the current prediction is greater than the previous one a long trade is initiated.

Considering the results in Tables \ref{tab:chp_ts:arima_ann_sys1} and \ref{tab:chp_ts:arima_ann_sys2} it can be seen that System 1 outperforms System 2 quite markedly. Even so, the results are quite modest across most of the indices and especially poor for the Dax. The results prove inferior to the baseline Naive Reversing System introduced in \ref{sec:naive:rev} Chapter \ref{Chapter4} as shown in Table \ref{tab:chp_ts:arima_ann_sys1_diff}.

%label - tab:chp_ts:arima_ann_sys1_diff
% latex table generated in R 3.1.0 by xtable 1.7-3 package
% Mon Aug 11 20:47:55 2014
\begin{table}[ht]
\centering
\caption[Arima/ANN predictions passed to System 1 compared to Naive Reversing methodology]{Results from a trading system based on forecasts of closing price generated by the Arima/ANN model compared to baseline Naive Reversing methodology.} 
\label{tab:chp_ts:arima_ann_sys1_diff}
\begin{tabular}{lcc}
  \toprule Mkt & Diff in Mean Long PL & Diff in Mean Short PL \\ 
  \midrule DAX & -2 & -4 \\ 
  CAC & 1 & -2 \\ 
  FTSE & -2 & -4 \\ 
  Dow & -5 & -12 \\ 
  Nikkei & 1 & -8 \\ 
  AORD & -6 & -1 \\ 
   \bottomrule \end{tabular}
\end{table}


\subsubsection{ARIMA/k-Nearest Neighbour (k-NN)}
An alternative to the ARIMA/ANN methodology is to replace ANN with a k-Nearest Neighbour learner. Results from using the forecasts generated in the two trading systems introduced in section \ref{sec:traing:arima:models} can be seen in Tables \ref{tab:chp_ts:pred_close_arima_knn_sys1} and \ref{tab:chp_ts:pred_close_arima_knn_sys2}. The results from System 1 are very good and exceed the baseline Naive Reversing approach. Table \ref{tab:chp_ts:pred_close_arima_knn_sys1_diff} lists the difference in results between those generated with System 1 and the ARIMA/k-NN models and the baseline system. In all cases the hybrid ARIMA model produces superior results.

%label - tab:chp_ts:pred_close_arima_knn_sys1_diff
% latex table generated in R 3.1.0 by xtable 1.7-3 package
% Mon Jun 23 18:28:01 2014
\begin{table}[ht]
\centering
\caption[Predicting Close Price - Arima/k-NN predictions passed to System 1.]{Predicting Close Price - Arima/k-NN predictions compared with Naive Reversing System.} 
\label{tab:chp_ts:pred_close_arima_knn_sys1_diff}
\begin{tabular}{lcccccc}
  \toprule Mkt & LongPL & ShortPL & L Win \% & Av L PL & S Win \% & Av S PL \\ 
  \midrule Dax & 7323 & 6769 & 3 & 3 & 3 & 4 \\ 
  CAC & 5344 & 4787 & 1 & 2 & 2 & 3 \\ 
  FTSE & 13321 & 12911 & 5 & 6 & 6 & 8 \\ 
  Dow & 14531 & 14502 & 3 & 7 & 4 & 9 \\ 
  Nikkei & 13050 & 12880 & 3 & 8 & 3 & 8 \\ 
  AORD & 6394 & 6401 & 4 & 3 & 5 & 4 \\ 
   \bottomrule \end{tabular}
\end{table}


\subsection{ARIMA Hybrids - Predicting Up Down with Categorical Label}
An alternative to forecasting the closing price of a financial market is to predict the general direction it will move in the short term either up or down. To this end an additional categorical label to indicate whether the market increased or fell in value over the course of the day was introduced into the data sets. This new attribute had the value  \textquotedblleft U" if the market increased and \textquotedblleft D" if it decreased. Hybrid ARIMA models were then employed to predict this label.

\subsubsection{ARIMA/Artificial Neural Networks (ANN)}
The first methodology employed was to combine ARIMA with Artificial Neural Networks (ANN) in order to generate a forecast of the categorical label that indicated whether the market increased in value or fell over the course of the day. Once the forecast was generated and added to the data set in the form of a new attribute it was passed to a trading algorithm which based the decision whether to trade long or short on the forecast generated. The R code for the trading algorithm can be seen in Appendix \ref{AppendixA} section \ref{appA:ts_4} and the results generated in Table \ref{tab:chp_ts:pUD_CAT_arima_ann_sys}. Overall the results were poor and inferior to the baseline system used for comparison. 

\subsubsection{ARIMA/k-Nearest Neighbour (k-NN)}
Replacing the ANN learner from the previous section with a k-NN method resulted in far better results.  Table \ref{tab:chp_ts:pUD_CAT_arima_knn_sys} lists the results of passing the forecasts from this combination to the trading algorithm in Appendix \ref{AppendixA} section \ref{appA:ts_4}. Across all the data sets large positive results are recorded. Table \ref{tab:chp_ts:pUD_CAT_arima_knn_sys_diff} lists the difference in results between using this hybrid ARIMA approach and the usual baseline returns. Clearly this methodology produces superior results. Using a stop loss with this system increases the returns from all the markets except the US Dow and these results are listed in Table \ref{tab:chp_ts:pUD_CAT_arima_knn_sys_SL}.

%label - tab:chp_ts:pUD_CAT_arima_knn_sys_diff
% latex table generated in R 3.1.0 by xtable 1.7-3 package
% Thu Aug 21 07:23:30 2014
\begin{table}[ht]
\centering
\caption[ARIMA/k-NN U/D system results minus Naive Reversing results]{PL from Naive Reversing system subtracted from results generated by a trading system based on ARIMA/k-NN U/D forecasts.} 
\label{tab:chp_ts:pUD_CAT_arima_knn_sys_diff}
\begin{tabular}{lcccc}
  \toprule Mkt & L Win \% & Av L PL & S Win \% & Av S PL \\ 
  \midrule DAX & 1 & -4 & -2 & -6 \\ 
  CAC & -1 & 0 & -4 & -2 \\ 
  FTSE & 2 & 0 & 0 & -1 \\ 
  Dow & -1 & -4 & -5 & -11 \\ 
  Nikkei & -1 & 5 & -2 & -4 \\ 
  AORD & 1 & 0 & 2 & 7 \\ 
   \bottomrule \end{tabular}
\end{table}


%\subsubsection{ARIMA / SVN}

\subsection{ARIMA Hybrids - Predicting Up Down with Numeric Label}
The final approach adopted was to represent whether a financial market moved up or down by using 1 to signify that the market moved up and 0 that it moved down. The implications of using a numeric value is that the forecasts were in a range between these two values. In such circumstances the trading algorithms picked long trades when the prediction were in the upper half of the range.

\subsubsection{ARIMA/Artificial Neural Networks (ANN)}
An hybrid approach using ARIMA and ANN was used to make one-step forecasts for the future direction of the market, either up (1) or down (0). Table \ref{tab:chp_ts:pUD_01_arima_ann_sys} in Chapter \ref{Chapter5} lists the results of passing the indice data sets augmented with the ARIMA/ANN forecasts to the trading algorithm listed in Appendix \ref{AppendixA} section \ref{appA:ts_3a}.  Overall the results are poor, especially for the Japanese Nikkei trading long and inferior to the Naive Reversing system that is used as a comparative baseline.

\subsubsection{ARIMA/k-Nearest Neighbour (k-NN)}
Finally a k-Nearest Neighbour (k-NN) learner was used instead of the ANN algorithm. The models were used to calculate the one-step ahead forecast as represented by a numeric value between 0 and 1. The forecast was added to the data sets and passed to the trading listed in Appendix \ref{AppendixA} section \ref{appA:ts_3}. In common with other forecast using the hybrid k-NN approach good results were obtained and these can be seen in Table \ref{tab:chp_ts:pUD_01_arima_knn_sys} of Chapter \ref{Chapter5}. The results are much better than the baseline system which simply trades based on doing the opposite of what happened yesterday. Table \ref{tab:chp_ts:pUD_01_arima_knn_sys_diff} lists the difference in terms of performance between the two systems and is simply the values in  Table \ref{tab:ntfresults} from Chapter \ref{Chapter4} subtracted from the results in Table \ref{tab:chp_ts:pUD_01_arima_knn_sys} from Chapter \ref{Chapter5}.

%label - tab:chp_ts:pUD_01_arima_knn_sys_diff
% latex table generated in R 3.1.0 by xtable 1.7-3 package
% Wed Jul 09 19:05:40 2014
\begin{table}[ht]
\centering
\caption[Predicting UpDn 01 - Arima/k-NN predictions passed to System 3.]{Results from Naive Reversing System subtracted from results generated from predicting Up/Down Numerical label using Arima/k-NN.} 
\label{tab:chp_ts:pUD_01_arima_knn_sys_diff}
\begin{tabular}{lcccccc}
  \toprule Mkt & LongPL & ShortPL & L Win \% & Av L PL & S Win \% & Av S PL \\ 
  \midrule Dax & 13175 & 12656 & 11 & 10 & 6 & 5 \\ 
  CAC & 10175 & 9730 & 12 & 9 & 4 & 3 \\ 
  FTSE & 13872 & 13448 & 12 & 11 & 6 & 6 \\ 
  Dow & 12307 & 12336 & 10 & 10 & 6 & 6 \\ 
  Nikkei & 19400 & 19063 & 12 & 24 & 3 & 3 \\ 
  AORD & 8343 & 8410 & 12 & 6 & 8 & 4 \\ 
   \bottomrule \end{tabular}
\end{table}




\section{Conclusion}
TA - not much cop -> b/out good

\section{Future Work}
candlestick systems -> price 2,,3,4 days ahead?
combining systems
k-NN seems promising
additional markets
