% Chapter 6

\chapter{Analysis} % Main chapter title

\label{Chapter6} % For referencing the chapter elsewhere, use \ref{Chapter6} 

\lhead{Chapter 6. \emph{Analysis}} % This is for the header on each page - perhaps a shortened title

%----------------------------------------------------------------------------------------

\section{Introduction}
In chapters 4 and 5 a wide variety of analytical techniques were applied to a range of time series data sets. In Chapter 4 a number of trading algorithms were developed based on technical analysis indicators, with the intention of automating the decision of whether to buy or sell a market. For comparison purposes, two simple so called "naive" systems were explored to act as a baseline against which the technical analysis indicators could be compared.  The technical indicators were grouped together into their general area of applicability, namely trend detection indicators, reversal, momentum and candlestick indicators.

Chapter 5 continued the exploration of financial time series through the use of exponential smoothing, ARIMA and hybrid ARIMA techniques. The generated models were used to create one-step ahead forecasts which were then added to the original data sets. These amended data sets were then fed into trading algorithms which used the forecast values to make buying decisions.

\section{Exponential Smoothing}
Exponential smoothing was used to make one step ahead forecasts for the indice data sets. Initially two base systems were explored in order that they could be used as a baseline against which later results could be compared. The two methods generated predictions using a mean method in which the forecast was simply the average of the sample and a drify method which is the extrapolation of a striaght line between the first and last point in a data sample. For all the forecasts generated in this section a moving window approach was adopted. A window of sample data points was used, typically the last 30 observations, for which a forecast was generated using one of the methods. This window was then advanced one observation forward and the forecast for this data set calculated and the process repeated for the entire data set.

Once the forecasts were generated they were passed to a trading algorithm which made decisions regarding whether to trade long or short based on the value of the forecast. If the forecast was higher than the previous closing price a long trade was initiated or if it was lower a short trade was made with the expectation that the market would fall. Results from the forecasts generated from the mean method can be seen in Figure \ref{} while results from the drift method can be seen in Figure \ref{}. Both systems produced poor results and wouldn't generate enough profits to offset any transaction costs that would be incurred in the real world.

Next exponential smoothing was used to  generate forecasts. Allowing for trend, seasonality and whether there are additive or multiplicative affects, a variety of models may be used to generate predictions. The ets() function of the forecast package in R can create models from all the available possibilities. Once again using a moving window approach, one step ahead forecasts were created for the indice data sets. As the window moved through the data, different models were selected as being the best fit and these used to generate the forecasts. Having generated the preditctions they were passed to the same trading algorithm as used for the mean and drift models, and the results can be seen in Figure \ref{}. 

RESULTS in diss are the same as mean model ...


\section{Technical Analysis}

\subsection{Baseline Systems}
Initially two simple, naive systems were explored so that they could be used as a baseline against which the developed predictive models could be compared. These systems were the Naive Long System which mirrors a buy and hold strategy and a Naive Reversing System which simply trades in the opposite direction to the previous days market movement.

The first baseline system tried was the Naive Long system in which a market buy is placed each day and is similar to the so-called \textquotedblleft Buy and Hold" technique. The assumption here is that the market rises over time and if an investor simply holds a security it will eventually generate a profit.  The total profit is the price at the start, in this case the data set started in 2000, subtracted from the price at the end of the period, which in this case was the end of 2013.

The first iteration of the algorithm placed a buy at the start of the trading session and closed it at the end and thus the system was out of the market overnight. This resulted in significant discrepancies from the returns expected from a buy and hold system. Table \ref{tab:ind_start_stop_chp6} lists the expected returns from a Buy and Hold system over this period, with the Difference column being the profit or loss over the time.

% ---------- Table
\begin{table}[!htbp] \centering  
\caption[Returns from a  \textquotedblleft Buy and Hold" technique]{Returns from a  \textquotedblleft Buy and Hold" technique.}
\label{tab:ind_start_stop_chp6}
\begin{tabular}{lcccc}
\toprule
Date & Start 2000 & End 2013 & Difference & \% Change  \\
\midrule
Dax & 6961 & 9552   & +2591 & +37 \\
CAC & 6024 & 4250   & -1774 & -29 \\
FTSE & 6930 & 6749  & -181  & +-3 \\
Dow & 11501 & 16576 & +5075 & +44 \\
Nik & 18937 & 16291 & -2646 & -14 \\
AORD & 3152 & 5353  & +2201 & +70 \\
\bottomrule
\end{tabular}
\end{table}

From simply trading long during market hours the Dax generated a loss as opposed to the 2591 profit expected, likewise the CAC showed a much larger loss than expected and the Nikkei resulted in a large loss when a small loss was expected. The Dow, FTSE and AORD were similar to the expected values. The discrepancies in the returns between the trading algorithm and a Buy and Hold approach was due to
the fact that the algorithm opened and closed trades each day as opposed to simply opening the trade and waiting several years. This first algorithm was simply trading within market hours, approximately 8am to 5pm local time, and was not in the market for the full 24 hours of the day.

Changing the algorithm such that the trades ran from the market close time until the close time of the following day and thus covered the full 24 hour period resulted in system results that matched those expected from a buy and hold approach.  Clearly the discrepancies from the first algorithm were due to the relative amounts the markets moved during the day as opposed to during the \textquotedblleft out of hours" trading. There is a slight bias for the markets to move upwards overnight and over the course of the study (14 years) adds up to significant values.  

The second naive system was termed the \textquotedblleft Naive Reversing" system and simply places a trade today in the opposite direction from the previous day. This idea produced reasonable returns, with every market making money. From these results it can be concluded that the markets have a tendency to "flip flop" and reverse back on themselves, and the phenomena of market reverses is well understood. This second concept produced far better results than the first and was thus used as the primary basis of comparison for the algorithms and systems developed from technical indicators and time series methods. For convenience the results from this system are reproduced in Table \ref{tab:n_rev_results_chp6}.

% \label{tab:n_rev_results_chp6}
% latex table generated in R 3.1.0 by xtable 1.7-3 package
% Fri Jul 11 07:22:22 2014
\begin{table}[ht]
\centering
\caption[Results from the Naive Reversing System.]{Results from a naive trading system which simply trades in the opposite direction to the previous day's movement.} 
\label{tab:n_rev_results_chp6}
\begin{tabular}{lcccccc}
  \toprule Mkt & LongPL & ShortPL & L Win \% & Av L PL & S Win \% & Av S PL \\ 
  \midrule Dax & 947 & 3131 & 53 & 1 & 49 & 2 \\ 
  CAC & 940 & 7810 & 53 & 1 & 53 & 4 \\ 
  FTSE & 4284 & 4115 & 53 & 3 & 50 & 2 \\ 
  Dow & 15799 & 6047 & 56 & 10 & 49 & 3 \\ 
  Nikkei & 2324 & 20486 & 51 & 1 & 54 & 12 \\ 
  AORD & 1264 & 237 & 53 & 1 & 48 & 0 \\ 
   \bottomrule \end{tabular}
\end{table}


%Naive Reversed PL:
%CAC - 7800 (L)
%FTSE - 4000 (LS)
%Dow - 6000/15800
%Nik - 20500(L)

\subsection{Trend Detection}
The first group of the technical analysis indicators studied were the trend detection indicators. Identification of trend direction and strength is very important in the world of financial trading and one of the most widely encountered phrases is "the trend is your friend", as most authorities advocate trading in the direction of the trend. (In fact on a recent webinar it was claimed that 80\% of all money made is made trading in the direction of the trend.)  Well known indicators that purport to assist the trader in identifying trends are the simple moving average (SMA), the moving average convergence/divergence indicator (MACD) and the Aroon indicator.

The use of simple moving average is wide-spread in the financial markets. Market participants track moving averages or even more than one and make a decision which way to trade based on the position of the current price relative to it. Popular values to use in the SMA are 25, 50 and 200 for the look back period. The results of a trading system based on SMA is presented in Table \ref{tab:sma_results} of Chapter \ref{Chapter4}. The algorithm places a buy trade if the current price is above the SMA and a sell trade if it is below it. 

Results are mixed from using the SMA, with some markets producing positive results and some ending in losses. The German Dax and Australian AORD produce positive results across all the SMA values with returns from trading short (predicting the market will decline) doing best.  The Japanese Nikkei and French CAC display different behaviour in that all the SMA values tried produce negative results in trying to  predict long trades but positive returns when attempting to predict short trades.  The UK's FTSE 100 is different again, producing negative results across the board. Finally, the Dow produces positive results for trades on the long side but losses for trading short.

%DOES this reflect on how these markets move? - smooth trends would be profitable ...
%NOTE - worth analysing "trending" behaviour?  Amount of time >70 in aroon?   

In an attempt to improve the returns from the trading system a stop loss was introduced. Comparing Tables \ref{tab:sma_results} and \ref{tab:sma_results_Sloss} of Chapter \ref{Chapter4}
it can be seen that applying the stop loss has been on the whole beneficial to the results obtained, with the exception of those from the Dow which were negatively impacted. Essentially losing trades have been truncated while winning trades have been left to develop. This general pattern of a stop loss being beneficial to all the markets except the US Dow was seen multiple times with the systems tested.

%b. MACD – no SL applied here
The second trend detection indicator explored was the Moving Average Convergence Divergence (MACD) indicator, full details of which can be found in section \ref{appB:MACD} of Appendix \ref{AppendixB}. The MACD can generally be used two ways, as a trend detection indicator and as an over-bought/over-sold indicator in which case traders use it to identify potential market reversals. In this section the indicator was used as a trend detector and the results from a system based on the MACD indicator can be seen in Table \ref{tab:mac_trend_results} in Chapter \ref{Chapter4}.  The algorithm trades long when the value of MACD is greater than the value of the signal line, see Appendix \ref{AppendixA} section \ref{appA:macd_xov} for details of the R code used. The results are not very impressive, only the Nikkei producing reasonable profits, although they wouldn't beat the baseline Naive Reversing system.

The final trend detection indicator examined was Aroon.  This indicator measures the number of intervals since the previous high or low within a certain time window. The algorithms presented here used a time window of 20 days. If the current day was the highest price in the last 20 days trading, the indicator would take a value of 100 and for each following day that doesn't make a new high the indicator falls by 5 (100 divided by the lag period which is 20).  Thus if the highest price was four days ago the AroonUp value would be 80. The opposite situation occurs with regard to the low price. A value of 70 or above for the AroonUp is indicative of a upward trending market and likewise a value of 70 and above for AroonDn suggests a falling market.

The results from an algorithm using the Aroon indicator can be seen in Table \ref{tab:aroon_results} of Chapter \ref{Chapter4}. Overall the results are encouraging with the Dax, FTSE, Dow and AORD all producing positive returns for both long and short trades, while the CAC and Nikkei are positive when trading short. Table \ref{tab:aroon_results_diff} lists the values derived from the Aroon system with those from the baseline Reversing system (see Chapter \ref{Chapter4} section \ref{sec:naive:rev}) subtracted. Because the Aroon system doesn't execute trades each day it only makes sense to compare the average daily returns as opposed to the total returns. As can be seen from  Table \ref{tab:aroon_results_diff}, compared with the baseline system for some markets the Aroon indicator outperforms the baseline while for others it is worse, notably the Nikkei. Considering the second column in Table \ref{tab:aroon_results_diff}, \textquotedblleft Diff in Mean Long PL" only the Dax and AORD outperformed the baseline reversing system in producing long winning trades. Alternatively, the system based on the Aroon indicator was superior in predicting winning short trades for all the markets except the Nikkei, as seen in the third column \textquotedblleft Diff in Mean Short PL".

%\ref{tab:aroon_results_diff}
% latex table generated in R 3.1.0 by xtable 1.7-3 package
% Fri May 30 19:26:05 2014
\begin{table}[ht]
\centering
\caption[Naive Long System - Close to Close]{Naive Long System changed such that the trading period is the previous close price minus today's close.} 
\label{tab:aroon_results_sloss_diff}
\begin{tabular}{lcc}
  \toprule Market & Long Difference & Short Difference \\ 
  \midrule Dax & 102 & 2208 \\ 
  CAC & 414 & 1167 \\ 
  F100 & 49 & 2300 \\ 
  Dow & -18053 & -13152 \\ 
  Nik & 8005 & 10164 \\ 
  Oz & 51 & 619 \\ 
   \bottomrule \end{tabular}
\end{table}


The trading system based on the Aroon indicator was re-run with a stop loss value of 100. Overall the use of a stop loss improves the returns, with the exception of the Dow. One again using a stop loss with the Dow shows very marked negative impacts on profits. These results can be seen in \ref{tab:aroon_results_sloss} of Chapter \ref{Chapter4}.

\subsection{Market Reversal Indicators}
In this section two indicators that purport to assist in identifying market reversals are examined, namely the Parabolic Stop-and-Reverse (SAR) and the Moving Average Convergence Divergence (MACD) used as an over-bought/over-sold indicator. 

The first market reversal indicator explored was the Parabolic Stop-and-Reverse (SAR), an indicator initially developed for traders who were always in the market with either long or short position. The SAR is used to judge when the  position should be reversed from long to short or vice versa. The trading algorithm reported here trades each day (i.e opens a trade at the start of the trading session and closes it out at the end) and makes a decision regarding the direction of the trade based on the SAR indicator. If the market opening is above the SAR a long trade is initiated and vice versa if the market is below the SAR value.

The results from the trading system based on the SAR can be seen in Table \ref{tab:sar_results} of Chapter \ref{Chapter4} and are very poor. Only the Nikkei trading short produces reasonable results, but these are much worse than the baseline Naive Reversing method introduced previously.

As previously mentioned the MACD indicator can be used as a market reversal indicator. Once the MACD value reaches its extreme values, the market is considered over-bought or over-sold and likely to reverse back on itself. The trading algorithm using this concept expects a market reversal once the MACD crosses above the 85\% quantile (of the MACD range) or below the 15\% quantile. Short trades are initiated once the MACD crosses above the 85\% quantile value and short trades once it has passed below the 15\% quantile. The results from this trading system can be seen in Table \ref{tab:mac_ob_results} and are very unimpressive being inferior to the baseline method.

\subsection{Momentum Indicators}
A third category of technical indicators are the momentum indicators, which are related to the trend detection indicators. Two such indicators are studied here, the Stochastic and Rate of Change (ROC). The stochastic oscillator is one of the oldest and most widely used of the technical indicators. It measures the percentage position the current close is in relation to the high low range of the period of interest. For example the current close could be 80\% of the way between the low and high of the last 10 days. Thus it has conceptual similarities to the Aroon indicator. The stochastic is usually represented by two lines \%K which is the position of the price within this high low envelope described above, and \%D a moving average of \%K (see Appendix \ref{AppendixB} section \ref{appB:stoch} for more details). 

The trading algorithm utilising the stochastic initiates long trades when \%K is above \%D and short trades when \%K is below \%D. Results from an algorithm implementing these ideas can be seen in  Table \ref{tab:stoch_results} in Chapter \ref{Chapter4}. The results of this system are poor being significantly worse than the baseline Naive Reversing system. 

%ROC1 - Mkt$Long <- ifelse(Mkt$prevROC < lw,Mkt$Close-Mkt$Open,NA)
%ROC2 - Mkt$Long <- ifelse(Mkt$prevROC > 0,Mkt$Close-Mkt$Open,NA)

The second momentum indicator is the Rate Of Change (ROC) indicator, and this is simply the difference between the current price and a price a certain number of days previously. If this value is positive the market is considered to be trending up and the larger the value the greater the trending momentum. The results from an algorithm using these ideas is presented in Table \ref{tab:stoch_results} of Chapter \ref{Chapter4}. The results are positive but very modest and inferior to the baseline Reversing system.

\subsection{Breakout systems}
The fourth area of technical analysis explored the idea of trade signals being generated by a particular value from the previous day, so-called breakout systems. Two particular values are used as the trigger price for a trade, the previous day's high/low or the 90\% quantile of the minor move (see section \ref{sec:ohol:fluctuation} of Chapter \ref{Chapter3}). 

The first idea explored was to use the previous time period's high or low price as a trigger for a buy or sell. If the current day's high price exceeded the previous day's high price a long trade would be made and in a similar manner if today's low price is lower than previous day's low a short trade is initiated. Results from using the previous day's high price or low price as a trigger to trade long or short can be seen in Table \ref{tab:hl_bout_sys}. Generally the results are very good with the exception of the Dow. These results can be linked to the data exploratory work shown in Table \ref{tab:closeHL} of section \ref{sec:closing_prices}. The best returns were generated in the Nikkei, a market which had the highest number of times closing outside the previous day's high or low. Conversely, the lowest ranked market in terms of closing outside yesterday's high low range was the Dow, and this was the one market that produced negative results in the break-out system. Table \ref{tab:hl_bout_sys_diff} lists the returns from the high low breakout system with the profits from the baseline Naive Reversing system subtracted. As can be seen, with the exception of the Dow, the method out-performs the baseline system markedly.

%lab = 'tab:hl_bout_sys_diff'
% latex table generated in R 3.1.0 by xtable 1.7-3 package
% Tue Jun 10 16:51:55 2014
\begin{table}[ht]
\centering
\caption[Daily High / Low Breakout System compared with Naive Reversing System]{Results from Daily High / Low Breakout System compared with Naive Reversing System} 
\label{ab:hl_bout_sys_diff}
\begin{tabular}{lcccccc}
  \toprule Mkt & LongPL & ShortPL & L Win \% & Av L PL & S Win \% & Av S PL \\ 
  \midrule Dax & 20126 & 18098 & 5 & 10 & 7 & 11 \\ 
  CAC & 13312 & 12366 & 5 & 7 & 6 & 8 \\ 
  FTSE & 8955 & 14499 & 6 & 4 & 9 & 10 \\ 
  Dow & -35154 & -33381 & -14 & -21 & -11 & -20 \\ 
  Nikkei & 72276 & 61159 & 13 & 43 & 10 & 37 \\ 
  AORD & 18083 & 21007 & 14 & 10 & 17 & 14 \\ 
   \bottomrule \end{tabular}
\end{table}


The second break-out system used the minor fluctuation 90\% quantile value as the trigger level to trade long or short. Once the market moved above this level a long trade was made or if the market moved below this level a short trade was executed. Overall this methodology produces good results with the exception of the Dow and CAC. Table \ref{tab:chp_ta_90q_diff} lists the difference in results between this breakout methodology and the baseline Naive Reversing system.

%lab = 'tab:chp_ta_90q_diff'
% latex table generated in R 3.1.0 by xtable 1.7-3 package
% Fri Jul 11 07:22:30 2014
\begin{table}[ht]
\centering
\caption[Daily 90\% Quantile level Breakout System compared with Naive Reversing System]{Results 90\% Quantile level Breakout System compared with Naive Reversing System} 
\label{tab:chp_ta_90q_diff}
\begin{tabular}{lcccccc}
  \toprule Mkt & LongPL & ShortPL & L Win \% & Av L PL & S Win \% & Av S PL \\ 
  \midrule Dax & 6894 & 3240 & 3 & 5 & 4 & 2 \\ 
  CAC & 1707 & -2725 & 1 & 1 & -1 & -1 \\ 
  FTSE & 6474 & 11180 & 3 & 4 & 4 & 8 \\ 
  Dow & -46061 & -40901 & -17 & -34 & -12 & -31 \\ 
  Nikkei & 21282 & 11344 & 7 & 15 & 2 & 8 \\ 
  AORD & 15466 & 19120 & 10 & 8 & 14 & 12 \\ 
   \bottomrule \end{tabular}
\end{table}


\subsection{Candlestick Patterns}
A number of so-called candlestick patterns were explored for predictive properties in financial markets. The patterns tested were essentially market reversal patterns. Firstly, Hammer and Inverted Hammer were considered. When these patterns occur it is considered a sign that the market will move upwards, especially when they are encountered in a down trend, thus reversing direction. Table \ref{tab:hammer_results} lists the results from placing buy trades after all occurrences of either pattern while Table \ref{tab:hammer_aroon_results} shows the results from initiating buy trades when these patterns occur in trending markets. The Aroon indicator detailed in section \ref{appB:aroon} of Appendix \ref{AppendixB} was used to determine if the market was in a trending phase. Overall the results from using the Hammer or Inverted Hammer candlestick pattern to predict market movement was poor. Only the Dow and FTSE showed positive results, although the per trade profit from the Dow was good. Another consideration is the small number of times in which these patterns occur, only 22 trades in the 14 years of the Dow data were made.  Clearly these visual patterns are quite subjective and in reality a trader would use judgement as to whether the pattern constituted a Hammer or not. However, in developing an algorithm to recognise and trade them no such latitude is possible and thus the number of trades taken by the algorithms is likely to be less than in reality.

The next pattern tested was the Engulfing pattern. This pattern occurs when a candlestick has a lower low and a higher high than the previous day's candlestick, it engulfs it. The presence of this pattern it supposed to indicate that the market will change direction. The results of a trading algorithm that trades long or short depending upon the presence of an Engulfing candlestick can be seen in Table \ref{tab:engulf_results}. The results shown in Table \ref{tab:engulf_aroon_results} are similar to Table \ref{tab:engulf_results} except trades are only taken if the market is trending, with the Aroon indicator used to determine if the market is in a trending phase. The results from both algorithms were very poor, with most markets showing negative results.

The final pattern tested was the Doji, one of the best known candlestick patterns. Again the presence of this pattern in a trending market is supposed to give warning to the market participants that a reversal may be imminent. Table \ref{tab:doji_aroon_results} shows the results of a trading system that uses the presence of a Doji in a trending market to initiate a trade. Again the results are very poor with mostly negative returns.

\section{Time Series Analysis}
ARIMA and hybrid ARIMA models were used to generate forecasts of the closing prices and the more general situation of whether the market would rise or fall. In modelling the more general situation of market direction a categorical and a continuous label was employed. The categorical label used \textquotedblleft U" to represent occasions when the market prices increased and \textquotedblleft D" for when it decreased in value. Alternatively the values 1 and 0 were also used to represent up and down respectively. The primary difference between the two labels was in the values returned from the hybrid ARIMA models. When using 1 and 0 for the class label the models return a value in the range of 1 to 0, whereas for the categorical value there was only the choice of the two values. 

\subsection{ARIMA Models}
The auto.arima() function of the R forecast package was used to assist in generating ARIMA models for the national indice data sets used in this study. For convenience the models selected are listed in Table \ref{tab:chp_ts_arima_models_chp6}.

%label - tab:chp_ts_arima_models
% latex table generated in R 3.1.0 by xtable 1.7-3 package
% Sat Aug 16 11:47:58 2014
\begin{table}[ht]
\centering
\caption[ARIMA models chosen for the indice data sets]{ARIMA models chosen to forecast future values in the national indice data sets.} 
\label{tab:chp_ts_arima_models_chp6}
\begin{tabular}{lc}
  \toprule Market & ARIMA Model \\ 
  \midrule DAX & ARIMA(3,1,3)                    \\ 
  CAC & ARIMA(2,1,3)                    \\ 
  FTSE & ARIMA(2,1,3)                    \\ 
  Dow & ARIMA(1,1,2)                    \\ 
  Nikkei & ARIMA(2,1,3)                    \\ 
  AORD & ARIMA(1,1,0)                    \\ 
   \bottomrule \end{tabular}
\end{table}


The one-step forecasts generated from these models were then used in two trading systems. In the first algorithm the decision to trade long or short was dependant upon on the relative values of the previous close price and the forecast. If the forecast was higher than the close price a long trade was entered in the expectation that the market would rise towards the prediction. The opposite situation was expected for when the forecast was lower than the close price. The R code for this first algorithm can be seen in Appendix \ref{AppendixA} section \ref{appA:ts_1} and is labelled system 1.

The second trading algorithm used the relative values of the predictions themselves in order to decide whether to trade long or short. If the current forecast was higher than the previous one a long trade was made and vice versa.  The R code for this second algorithm can be seen in Appendix \ref{AppendixA} section \ref{appA:ts_2} and is labelled system 2.

The results from both systems were poor. The difference in mean PL per trade between the first system based on the auto.arima models (previous close in comparison to forecast) and the mean PL for the Naive Reversing system from section \ref{sec:naive:rev} Chapter \ref{Chapter4} can be seen in Table \ref{tab:chp_ts:arima1_diff}. Most of the results are worse than the naive baseline system except for the French CAC and US Dow when trading long.

%label - tab:chp_ts:arima1_diff
% latex table generated in R 3.1.0 by xtable 1.7-3 package
% Sat Aug 23 08:35:25 2014
\begin{table}[ht]
\centering
\caption[ARIMA system results minus Naive Reversing results]{PL from Naive Reversing system subtracted from results generated by a trading system based on ARIMA forecasts.} 
\label{tab:chp_ts:arima1_diff}
\begin{tabular}{lcc}
  \toprule Mkt & Diff in Mean Long PL & Diff in Mean Short PL \\ 
  \midrule DAX & -6 & -11 \\ 
  CAC & 2 & -3 \\ 
  FTSE & 1 & -3 \\ 
  Dow & -3 & -14 \\ 
  Nikkei & 20 & 1 \\ 
  AORD & 1 & -1 \\ 
   \bottomrule \end{tabular}
\end{table}


\subsection{ARIMA Hybrids - Predicting Closing Price}
Hybrid ARIMA models in which Artificial Neural Networks and k-Nearest Neighbour algorithms were used instead of regression in the ARIMA algorithm were used to predict the closing prices of financial markets, see Chapter \ref{Chapter5} section \ref{sec:arima:chp5} for details.

\subsubsection{ARIMA/Artificial Neural Networks (ANN)}
Overall the use of the models generated from hybrid ARIMA/ANN algorithms to create trading systems was not very successful. The results from passing the indice data sets augmented with a forecast attribute generated by the hybrid ARIMA models can be seen in Tables  \ref{tab:chp_ts:arima_ann_sys1} and \ref{tab:chp_ts:arima_ann_sys2} of Chapter \ref{Chapter5}. System 1 compares the price of the forecast with the price of the previous close and in the event that the prediction is higher than the previous closing price a long trade is entered. The opposite is true when the forecast is lower than the closing price and a short trade is made. System 2 is similar but compares the forecast with the last forecast. In the event that the current prediction is greater than the previous one a long trade is initiated.

Considering the results in Tables \ref{tab:chp_ts:arima_ann_sys1} and \ref{tab:chp_ts:arima_ann_sys2} it can be seen that System 1 outperforms System 2 quite markedly. Even so, the results are quite modest across most of the indices and especially poor for the Dax. The results prove inferior to the baseline Naive Reversing System introduced in \ref{sec:naive:rev} Chapter \ref{Chapter4} as shown in Table \ref{tab:chp_ts:arima_ann_sys1_diff}.

%label - tab:chp_ts:arima_ann_sys1_diff
% latex table generated in R 3.1.0 by xtable 1.7-3 package
% Tue Jul 22 20:47:42 2014
\begin{table}[ht]
\centering
\caption[Arima/ANN predictions passed to System 1 compared to Naive Reversing methodology]{Results from a trading system based on forecasts of closing price generated by the Arima/ANN model compared to baseline Naive Reversing methodology.} 
\label{tab:chp_ts:arima_ann_sys1_diff}
\begin{tabular}{lcc}
  \toprule Mkt & Diff in Mean Long PL & Diff in Mean Short PL \\ 
  \midrule Dax & -1 & 323 \\ 
  CAC & -1 & 1 \\ 
  FTSE & 29 & -2 \\ 
  Dow & -6 & 1 \\ 
  Nikkei & 3 & -6 \\ 
  AORD & 2 & 0 \\ 
   \bottomrule \end{tabular}
\end{table}


\subsubsection{ARIMA/k-Nearest Neighbour (k-NN)}
An alternative to the ARIMA/ANN methodology is to replace ANN with a k-Nearest Neighbour learner. Results from using the forecasts generated in the two trading systems introduced in section \ref{sec:traing:arima:models} can be seen in Tables \ref{tab:chp_ts:pred_close_arima_knn_sys1} and \ref{tab:chp_ts:pred_close_arima_knn_sys2}. The results from System 1 are very good and exceed the baseline Naive Reversing approach. Table \ref{tab:chp_ts:pred_close_arima_knn_sys1_diff} lists the difference in results between those generated with System 1 and the ARIMA/k-NN models and the baseline system. In all cases the hybrid ARIMA model produces superior results.

%label - tab:chp_ts:pred_close_arima_knn_sys1_diff
% latex table generated in R 3.1.0 by xtable 1.7-3 package
% Sat Aug 16 11:48:15 2014
\begin{table}[ht]
\centering
\caption[Mean PL from hybrid ARIMA/k-NN models minus mean PL from Naive Reverse system]{Results from a system using forecasts from a ARIMA/k-NN model with the results of the Naive Reversing System subtracted.} 
\label{tab:chp_ts:pred_close_arima_knn_sys1_diff}
\begin{tabular}{lcccc}
  \toprule Mkt & L Win \% & Av L PL & S Win \% & Av S PL \\ 
  \midrule DAX & -1 & -1 & -4 & -2 \\ 
  CAC & -1 & -2 & -3 & -3 \\ 
  FTSE & 1 & -2 & 0 & -3 \\ 
  Dow & 1 & 1 & -3 & -7 \\ 
  Nikkei & -3 & -1 & -4 & -10 \\ 
  AORD & 0 & 0 & 2 & 7 \\ 
   \bottomrule \end{tabular}
\end{table}


\subsection{ARIMA Hybrids - Predicting Up Down with Categorical Label}
An alternative to forecasting the closing price of a financial market is to predict the general direction it will move in the short term either up or down. To this end an additional categorical label to indicate whether the market increased or fell in value over the course of the day was introduced into the data sets. This new attribute had the value  \textquotedblleft U" if the market increased and \textquotedblleft D" if it decreased. Hybrid ARIMA models were then employed to predict this label.

\subsubsection{ARIMA/Artificial Neural Networks (ANN)}
The first methodology employed was to combine ARIMA with Artificial Neural Networks (ANN) in order to generate a forecast of the categorical label that indicated whether the market increased in value or fell over the course of the day. Once the forecast was generated and added to the data set in the form of a new attribute it was passed to a trading algorithm which based the decision whether to trade long or short on the forecast generated. The R code for the trading algorithm can be seen in Appendix \ref{AppendixA} section \ref{appA:ts_4} and the results generated in Table \ref{tab:chp_ts:pUD_CAT_arima_ann_sys}. Overall the results were poor and inferior to the baseline system used for comparison. 

\subsubsection{ARIMA/k-Nearest Neighbour (k-NN)}
Replacing the ANN learner from the previous section with a k-NN method resulted in far better results.  Table \ref{tab:chp_ts:pUD_CAT_arima_knn_sys} lists the results of passing the forecasts from this combination to the trading algorithm in Appendix \ref{AppendixA} section \ref{appA:ts_4}. Across all the data sets large positive results are recorded. Table \ref{tab:chp_ts:pUD_CAT_arima_knn_sys_diff} lists the difference in results between using this hybrid ARIMA approach and the usual baseline returns. Clearly this methodology produces superior results. Using a stop loss with this system increases the returns from all the markets except the US Dow and these results are listed in Table \ref{tab:chp_ts:pUD_CAT_arima_knn_sys_SL}.

%label - tab:chp_ts:pUD_CAT_arima_knn_sys_diff
% latex table generated in R 3.1.0 by xtable 1.7-3 package
% Tue Jul 22 20:47:45 2014
\begin{table}[ht]
\centering
\caption[Predicting UpDn CAT - Arima/k-NN predictions passed to System 4 - ]{Results from Naive Reversing System subtracted from results generated from predicting Up/Down categorical label using Arima/k-NN.} 
\label{tab:chp_ts:pUD_CAT_arima_knn_sys_diff}
\begin{tabular}{lcccccc}
  \toprule Mkt & LongPL & ShortPL & L Win \% & Av L PL & S Win \% & Av S PL \\ 
  \midrule Dax & 14745 & 14226 & 8 & 7 & 11 & 10 \\ 
  CAC & 9221 & 8777 & 7 & 5 & 6 & 5 \\ 
  FTSE & 11269 & 10845 & 7 & 5 & 10 & 8 \\ 
  Dow & 14548 & 14577 & 6 & 4 & 11 & 12 \\ 
  Nikkei & 24882 & 24545 & 9 & 17 & 6 & 12 \\ 
  AORD & 8447 & 8514 & 7 & 4 & 11 & 6 \\ 
   \bottomrule \end{tabular}
\end{table}


\subsubsection{ARIMA / SVN}

%label - tab:chp_ts:pUD_CAT_arima_svm_sys
% latex table generated in R 3.1.0 by xtable 1.7-3 package
% Tue Aug 19 13:19:33 2014
\begin{table}[ht]
\centering
\caption[Results from a trading system using the forecast of categorical label "U/D" from hybrid ARIMA/SVM model]{Results from a trading system using the forecast of categorical label "U/D" from hybrid ARIMA/SVM model.} 
\label{tab:chp_ts:pUD_CAT_arima_svm_sys}
\begin{tabular}{lcccccc}
  \toprule Mkt & LongPL & ShortPL & L Win \% & Av L PL & S Win \% & Av S PL \\ 
  \midrule DAX & -123 & -322 & 53 & 0 & 46 & -1 \\ 
  CAC & -1607 & -612 & 49 & -4 & 47 & -1 \\ 
  FTSE & 2115 & 866 & 54 & 5 & 49 & 1 \\ 
  Dow & 2138 & -4686 & 57 & 10 & 45 & -6 \\ 
  Nikkei & 9 & 1135 & 48 & 0 & 51 & 2 \\ 
  AORD & -2364 & 301 & 53 & -4 & 49 & 1 \\ 
   \bottomrule \end{tabular}
\end{table}


%\subsection{ARIMA Hybrids - Predicting Up Down with Numeric Label}
%The final approach adopted was to represent whether a financial market moved up or down by using 1 to signify that the market moved up and 0 that it moved down. The implications of using a numeric value is that the forecasts were in a range between these two values. In such circumstances the trading algorithms picked long trades when the prediction were in the upper half of the range.
%
%\subsubsection{ARIMA/Artificial Neural Networks (ANN)}
%An hybrid approach using ARIMA and ANN was used to make one-step forecasts for the future direction of the market, either up (1) or down (0). Table \ref{tab:chp_ts:pUD_01_arima_ann_sys} in Chapter \ref{Chapter5} lists the results of passing the indice data sets augmented with the ARIMA/ANN forecasts to the trading algorithm listed in Appendix \ref{AppendixA} section \ref{appA:ts_3a}.  Overall the results are poor, especially for the Japanese Nikkei trading long and inferior to the Naive Reversing system that is used as a comparative baseline.
%
%\subsubsection{ARIMA/k-Nearest Neighbour (k-NN)}
%Finally a k-Nearest Neighbour (k-NN) learner was used instead of the ANN algorithm. The models were used to calculate the one-step ahead forecast as represented by a numeric value between 0 and 1. The forecast was added to the data sets and passed to the trading listed in Appendix \ref{AppendixA} section \ref{appA:ts_3}. In common with other forecast using the hybrid k-NN approach good results were obtained and these can be seen in Table \ref{tab:chp_ts:pUD_01_arima_knn_sys} of Chapter \ref{Chapter5}. The results are much better than the baseline system which simply trades based on doing the opposite of what happened yesterday. Table \ref{tab:chp_ts:pUD_01_arima_knn_sys_diff} lists the difference in terms of performance between the two systems and is simply the values in  Table \ref{tab:ntfresults} from Chapter \ref{Chapter4} subtracted from the results in Table \ref{tab:chp_ts:pUD_01_arima_knn_sys} from Chapter \ref{Chapter5}.
%
%%label - tab:chp_ts:pUD_01_arima_knn_sys_diff
%% latex table generated in R 3.1.0 by xtable 1.7-3 package
% Mon Jul 07 19:18:18 2014
\begin{table}[ht]
\centering
\caption[Predicting UpDn 01 - Arima/k-NN predictions passed to System 3.]{Results from Naive Reversing System subtracted from results generated from predicting Up/Down Numerical label using Arima/k-NN.} 
\label{tab:chp_ts:pUD_01_arima_knn_sys_diff}
\begin{tabular}{lcccccc}
  \toprule Mkt & LongPL & ShortPL & L Win \% & Av L PL & S Win \% & Av S PL \\ 
  \midrule Dax & 13175 & 12656 & 11 & 10 & 6 & 5 \\ 
  CAC & 10175 & 9730 & 12 & 9 & 4 & 3 \\ 
  FTSE & 13872 & 13448 & 12 & 11 & 6 & 6 \\ 
  Dow & 12307 & 12336 & 10 & 10 & 6 & 6 \\ 
  Nikkei & 19400 & 19063 & 12 & 24 & 3 & 3 \\ 
  AORD & 8343 & 8410 & 12 & 6 & 8 & 4 \\ 
   \bottomrule \end{tabular}
\end{table}




\section{Conclusion}

TA - not much cop -> b/out good, trend Det - Aroon OK, Rev no good ...

2.1.4 - MACD - reported profitability - not me

2.1.5 - Candlesticks - Marshall reported that they don't work, me same, but visual inspection suggests they do ...

\section{Future Work}
candlestick systems -> price 2,,3,4 days ahead?
combining systems
k-NN seems promising
additional markets
