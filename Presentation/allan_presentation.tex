%%%%%%%%%%%%%%%%%%%%%%%%%%%%%%%%%%%%%%%%%
% Beamer Presentation
% LaTeX Template
% Version 1.0 (10/11/12)
%
% This template has been downloaded from:
% http://www.LaTeXTemplates.com
%
% License:
% CC BY-NC-SA 3.0 (http://creativecommons.org/licenses/by-nc-sa/3.0/)
%
%%%%%%%%%%%%%%%%%%%%%%%%%%%%%%%%%%%%%%%%%

%----------------------------------------------------------------------------------------
%	PACKAGES AND THEMES
%----------------------------------------------------------------------------------------

\documentclass{beamer}

\mode<presentation> {

% The Beamer class comes with a number of default slide themes
% which change the colors and layouts of slides. Below this is a list
% of all the themes, uncomment each in turn to see what they look like.

%\usetheme{default}
%\usetheme{AnnArbor}
%\usetheme{Antibes}
%\usetheme{Bergen}
%\usetheme{Berkeley}
%\usetheme{Berlin}
%\usetheme{Boadilla}
%\usetheme{CambridgeUS}
%\usetheme{Copenhagen}
%\usetheme{Darmstadt}
%\usetheme{Dresden}
%\usetheme{Frankfurt}
%\usetheme{Goettingen}
%\usetheme{Hannover}
%\usetheme{Ilmenau}
%\usetheme{JuanLesPins}
%\usetheme{Luebeck}
\usetheme{Madrid}
%\usetheme{Malmoe}
%\usetheme{Marburg}
%\usetheme{Montpellier}
%\usetheme{PaloAlto}
%\usetheme{Pittsburgh}
%\usetheme{Rochester}
%\usetheme{Singapore}
%\usetheme{Szeged}
%\usetheme{Warsaw}

% As well as themes, the Beamer class has a number of color themes
% for any slide theme. Uncomment each of these in turn to see how it
% changes the colors of your current slide theme.

%\usecolortheme{albatross}
%\usecolortheme{beaver}
%\usecolortheme{beetle}
%\usecolortheme{crane}
%\usecolortheme{dolphin}
%\usecolortheme{dove}
%\usecolortheme{fly}
%\usecolortheme{lily}
%\usecolortheme{orchid}
%\usecolortheme{rose}
%\usecolortheme{seagull}
%\usecolortheme{seahorse}
%\usecolortheme{whale}
%\usecolortheme{wolverine}

%\setbeamertemplate{footline} % To remove the footer line in all slides uncomment this line
%\setbeamertemplate{footline}[page number] % To replace the footer line in all slides with a simple slide count uncomment this line

%\setbeamertemplate{navigation symbols}{} % To remove the navigation symbols from the bottom of all slides uncomment this line
}

\usepackage{graphicx} % Allows including images
\usepackage{booktabs} % Allows the use of \toprule, \midrule and \bottomrule in tables

%----------------------------------------------------------------------------------------
%	TITLE PAGE
%----------------------------------------------------------------------------------------

\title[Predictions in Financial Time Series]{Predictions in Financial Time Series} % The short title appears at the bottom of every slide, the full title is only on the title page

\author{Allan Steel} % Your name
\institute[ITB] % Your institution as it will appear on the bottom of every slide, may be shorthand to save space
{
Institute of Technology Blanchardstown \\ % Your institution for the title page
\medskip
\textit{allan@allansteel.com} % Your email address
}
\date{\today} % Date, can be changed to a custom date

\begin{document}

\begin{frame}
\titlepage % Print the title page as the first slide
\end{frame}

\begin{frame}
\frametitle{Contents} % Table of contents slide, comment this block out to remove it
\tableofcontents % Throughout your presentation, if you choose to use \section{} and \subsection{} commands, these will automatically be printed on this slide as an overview of your presentation
\end{frame}

%----------------------------------------------------------------------------------------
%	PRESENTATION SLIDES
%----------------------------------------------------------------------------------------

%------------------------------------------------
\section{Introduction} %
%------------------------------------------------

%\subsection{Subsection Example} % A subsection can be created just before a set of slides with a common theme to further break down your presentation into chunks

\begin{frame}
\frametitle{Introduction}
\begin{itemize}
\item Technical Analysis
\item Time Series Analysis
\end{itemize}

\end{frame}


%------------------------------------------------
\section{Data} %
%------------------------------------------------

\begin{frame}
\Huge{\centerline{Data}}
\end{frame}

% ---- Frame ------------
\begin{frame}
\frametitle{Data}
\begin{itemize}
\item Financial Data - time series
\item Yahoo
\item National Indices - geographical spread
\item UK, Germany, France, US, Japan, Australia
\end{itemize}
\end{frame}

% ---- Frame ------------
\begin{frame}
\frametitle{OHLC Data}
\begin{figure}
\centering
\includegraphics[width=6cm]{../Figures/chp3_ohlc}
\caption[]{A schematic representation of open, high, low and closing prices (OHLC)}
\label{fig:chp3_ohlc}
\end{figure}
\end{frame}

% ---- Frame ------------
\begin{frame}
\frametitle{German Dax}
% latex table generated in R 3.1.0 by xtable 1.7-3 package
% Mon Aug 11 19:40:42 2014
\begin{table}[ht]
\centering
\caption[Final 6 rows of the DAX data set.]{Final 6 rows of the DAX data set} 
\label{tab:daxtail}
\begin{tabular}{lcccc}
  \toprule Date & Open & High & Low & Close \\ 
  \midrule 16/12/2013 & 9005 & 9188 & 8998 & 9164 \\ 
  17/12/2013 & 9143 & 9162 & 9085 & 9085 \\ 
  18/12/2013 & 9145 & 9191 & 9122 & 9182 \\ 
  19/12/2013 & 9280 & 9352 & 9257 & 9336 \\ 
  20/12/2013 & 9371 & 9413 & 9353 & 9400 \\ 
  23/12/2013 & 9400 & 9400 & 9400 & 9400 \\ 
   \bottomrule \end{tabular}
\end{table}

\end{frame}

% ---- Frame ------------
\begin{frame}
\frametitle{German Dax Summary Statistics}
\begin{table}[!htbp] \centering
\caption[Dax summary statistics.]{Summary statistics of the Dax data set.}
\label{tab:daxsum}
\begin{tabular}{lccccc}
\toprule
Statistic & N & Mean & St. Dev & Min & Max \\
\midrule
Open  & 3,621 & 5,858.36 & 1,559.40 & 2,203.97 & 9,752.11 \\
High  & 3,621 & 5,906.70 & 1,561.17 & 2,319.65 & 9,794.05 \\
Low   & 3,621 & 5,804.85 & 1,557.49 & 2,188.75 & 9,714.02 \\
Close & 3,621 & 5,857.74 & 1,559.39 & 2,202.96 & 9,742.96 \\
\bottomrule
%\normalsize
\end{tabular}
\end{table}
\end{frame}

% ---- Frame ------------
\begin{frame}
\frametitle{German Dax 2000 to 2013}
\begin{figure}
\centering
\includegraphics[width=10cm]{../Figures/chp3_dax_2000_2013}
\caption{Graph of German Dax in 2013.}
\label{fig:chp3_dax_2000_2013}
\end{figure}

\end{frame}

%------------------------------------------------
\section{Technical Analysis} %
%------------------------------------------------

\begin{frame}
\Huge{\centerline{Technical Analysis}}
\end{frame}

% ---- Frame ------------
\begin{frame}
\frametitle{Technical Analysis}
\begin{itemize}
\item Technical analysis is the study of historical prices
\item Practitioners of technical analysis in the past were referred to as chartists
\item all that was needed to know about a particular market was contained in its pricing chart
\end{itemize}

\end{frame}

% ---- Frame ------------
\begin{frame}
\frametitle{Technical Analysis}

\textit{\textquotedblleft Obviously I am biased against the chartist. This is not only a personal predilection, but a professional one as well. Technical Analysis is anathema to the academic world. We love to pick on it. Our bullying tactics are prompted by two considerations: (1) the method is patently false; and (2) it's easy to pick on. And while it may seem a bit unfair to pick on such a sorry target, just remember: it is your money we are trying to save.\textquotedblright}
%\footnotesize{
\begin{thebibliography}{99} % Beamer does not support BibTeX so references must be inserted manually as below
\bibitem[Malkiel, 1999]{p1} Malkiel, B.G. (1999)
\newblock A Random Walk Down Wall Street: Including a Life-cycle Guide to Personal Investing
\end{thebibliography}

\end{frame}

% ---- Frame ------------
\begin{frame}
\frametitle{Technical Indicators}

\begin{itemize}
\item Moving Average Convergence Divergence (MACD)
\item Aroon
\item Stochastic
\item Rate of Change (ROC)
\item Candlesticks
\end{itemize}

\end{frame}

% ---- Frame ------------
\begin{frame}
\frametitle{Moving Average Convergence Divergence (MACD)}


\end{frame}

%------------------------------------------------
\section{Time Series} %
%------------------------------------------------

\begin{frame}
\Huge{\centerline{Time Series}}
\end{frame}

% ---- Frame ------------
\begin{frame}
\frametitle{Time Series}

\begin{itemize}
\item ARIMA
\item Hybrid ARIMA
\end{itemize}

\end{frame}

% ---- Frame ------------
\begin{frame}
\frametitle{ARIMA}

\begin{itemize}
\item Plot the data to get a general feel for the time series and to establish if it is stationary.
\item Stabilize any variance in the data with a transformation process such as the Box-Cox method.
\item Arima models work with stationary data, so if necessary, take differences of the data until it is stationary.
\item Examine the auto-correlation and partial auto-correlation (ACF/PACF) plots in order to determine if an AR(p) or MA(q) model is appropriate.
\item Test the chosen model(s), using the AICc to determine if a better model is available.
\item Check the residuals from the best model by plotting the ACF, and doing a portmanteau test on them. If the results from these tests do not look like white noise, a modified model may be required.
\item Finally, once the residuals have a similar pattern to white noise, the model can be used to generate forecasts.
\end{itemize}

\end{frame}

%------------------------------------------------
\section{Results} %
%------------------------------------------------

\begin{frame}
\Huge{\centerline{Results}}
\end{frame}

% ---- Frame ------------
\begin{frame}
\frametitle{Results - Baseline Buy and Hold}

% latex table generated in R 3.1.0 by xtable 1.7-3 package
% Mon Jun 02 13:01:38 2014
\begin{table}[ht]
\centering
\caption[Naive Long System - Close to Close]{Naive Long System changed such that the trading period is the previous close price minus today's close.} 
\label{tab:nlng_results_2}
\begin{tabular}{lccc}
  \toprule Mkt & LongPL & L Win \% & L Trades \\ 
  \midrule Dax & 2649 & 53 & 3527 \\ 
  CAC & -1667 & 51 & 3585 \\ 
  F100 & 86 & 51 & 3531 \\ 
  Dow & 5219 & 53 & 3520 \\ 
  Nik & -2712 & 51 & 3437 \\ 
  Oz & 2229 & 53 & 3547 \\ 
   \bottomrule \end{tabular}
\end{table}


\end{frame}

% ---- Frame ------------
\begin{frame}
\frametitle{Results - Baseline Daily Reversal}

% latex table generated in R 3.1.0 by xtable 1.7-3 package
% Wed Jun 04 22:13:23 2014
\begin{table}[ht]
\centering
\caption[Naive Following System.]{Naive system which reverses the previous day's trade direction.} 
\label{tab:ntfresults}
\begin{tabular}{lcccccc}
  \toprule Mkt & LongPL & ShortPL & L Win \% & Av L PL & S Win \% & Av S PL \\ 
  \midrule Dax & 947 & 3131 & 53 & 1 & 49 & 2 \\ 
  CAC & 940 & 7810 & 53 & 1 & 53 & 4 \\ 
  F100 & 4284 & 4115 & 53 & 3 & 50 & 2 \\ 
  Dow & 15799 & 6047 & 56 & 10 & 49 & 3 \\ 
  Nik & 2324 & 20486 & 51 & 1 & 54 & 12 \\ 
  Oz & 1264 & 237 & 53 & 1 & 48 & 0 \\ 
   \bottomrule \end{tabular}
\end{table}


\end{frame}

% ---- Frame ------------
\begin{frame}
\frametitle{Results - Aroon Technical Indicator}

% latex table generated in R 3.1.0 by xtable 1.7-3 package
% Fri May 30 19:26:05 2014
\begin{table}[ht]
\centering
\caption[Naive Long System - Close to Close]{Naive Long System changed such that the trading period is the previous close price minus today's close.} 
\label{tab:aroon_results}
\begin{tabular}{lcccccc}
  \toprule Mkt & LongPL & ShortPL & L Win \% & L Trades & S Win \% & S Trades \\ 
  \midrule Dax & 5308 & 5257 & 56 & 1719 & 51 & 1178 \\ 
  CAC & -1638 & 4919 & 50 & 1703 & 52 & 1247 \\ 
  F100 & 3042 & 5715 & 52 & 1612 & 51 & 1156 \\ 
  Dow & 12131 & 3811 & 55 & 1745 & 49 & 1172 \\ 
  Nik & -4852 & 12013 & 49 & 1448 & 52 & 1248 \\ 
  Oz & 3735 & 3540 & 55 & 1791 & 50 & 1110 \\ 
   \bottomrule \end{tabular}
\end{table}


\end{frame}

% ---- Frame ------------
\begin{frame}
\frametitle{Results - Break-out Indicator}

% latex table generated in R 3.1.0 by xtable 1.7-3 package
% Mon Jun 02 13:01:48 2014
\begin{table}[ht]
\centering
\caption[Daily High / Low Breakout System]{Results from Daily High / Low Breakout System.} 
\label{tab:hl_bout_sys}
\begin{tabular}{lcccccc}
  \toprule Mkt & LongPL & ShortPL & L Win \% & L Trades & S Win \% & S Trades \\ 
  \midrule Dax & 21073 & 21229 & 58 & 1853 & 56 & 1611 \\ 
  CAC & 14252 & 20176 & 58 & 1860 & 59 & 1721 \\ 
  F100 & 13239 & 18614 & 59 & 1771 & 59 & 1578 \\ 
  Dow & -19355 & -27334 & 42 & 1799 & 38 & 1635 \\ 
  Nik & 74600 & 81645 & 64 & 1714 & 64 & 1651 \\ 
  Oz & 19347 & 21244 & 67 & 1753 & 65 & 1472 \\ 
   \bottomrule \end{tabular}
\end{table}


\end{frame}

% ---- Frame ------------
\begin{frame}
\frametitle{Results - ARIMA}

% latex table generated in R 3.1.0 by xtable 1.7-3 package
% Mon May 26 18:59:53 2014
\begin{table}[ht]
\centering
\caption[arima.]{ts1 arima.} 
\label{tab:chp_ts:arima1}
\begin{tabular}{lcccccc}
  \toprule Mkt & LongPL & ShortPL & L Win \% & Av L PL & S Win \% & Av S PL \\ 
  \midrule Dax & -10960 & -12171 & 16 & -46 & 14 & -43 \\ 
  CAC & -7051 & -7861 & 20 & -25 & 16 & -26 \\ 
  F100 & -9498 & -10728 & 5 & -39 & 5 & -37 \\ 
  Dow & -16405 & -21297 & 3 & -72 & 6 & -72 \\ 
  Nik & -15664 & -16840 & 22 & -76 & 22 & -73 \\ 
  Oz & -6720 & -7556 & 2 & -26 & 3 & -26 \\ 
   \bottomrule \end{tabular}
\end{table}


\end{frame}

% ---- Frame ------------
\begin{frame}
\frametitle{Results - Hybrid ARIMA / knn}

% latex table generated in R 3.1.0 by xtable 1.7-3 package
% Mon Jun 23 18:28:00 2014
\begin{table}[ht]
\centering
\caption[Results from passing closing price predictions from hybrid ARIMA/k-NN model to System 1]{Results from passing closing price predictions from hybrid ARIMA/k-NN model to System 1.} 
\label{tab:chp_ts:pred_close_arima_knn_sys1}
\begin{tabular}{lcccccc}
  \toprule Mkt & LongPL & ShortPL & L Win \% & Av L PL & S Win \% & Av S PL \\ 
  \midrule Dax & 8270 & 9900 & 56 & 4 & 52 & 6 \\ 
  CAC & 6284 & 12597 & 54 & 3 & 55 & 7 \\ 
  FTSE & 17605 & 17026 & 58 & 9 & 56 & 10 \\ 
  Dow & 30330 & 20549 & 59 & 17 & 53 & 12 \\ 
  Nikkei & 15374 & 33366 & 54 & 9 & 57 & 20 \\ 
  AORD & 7658 & 6638 & 57 & 4 & 53 & 4 \\ 
   \bottomrule \end{tabular}
\end{table}


\end{frame}

% ---- Frame ------------
\begin{frame}
\frametitle{Results - Hybrid ARIMA / knn Categorical}

% latex table generated in R 3.1.0 by xtable 1.7-3 package
% Thu Jul 10 20:51:13 2014
\begin{table}[ht]
\centering
\caption[Results from a trading system using the forecast of categorical label "U/D" from hybrid ARIMA/k-NN model]{Results from a trading system using the forecast of categorical label "U/D" from hybrid ARIMA/k-NN model.} 
\label{tab:chp_ts:pUD_CAT_arima_knn_sys}
\begin{tabular}{lcccccc}
  \toprule Mkt & LongPL & ShortPL & L Win \% & Av L PL & S Win \% & Av S PL \\ 
  \midrule Dax & 15692 & 17357 & 61 & 8 & 60 & 12 \\ 
  CAC & 10161 & 16587 & 60 & 6 & 59 & 9 \\ 
  FTSE & 15553 & 14960 & 60 & 8 & 60 & 10 \\ 
  Dow & 30347 & 20624 & 62 & 14 & 60 & 15 \\ 
  Nikkei & 27206 & 45031 & 60 & 18 & 60 & 24 \\ 
  AORD & 9711 & 8751 & 60 & 5 & 59 & 6 \\ 
   \bottomrule \end{tabular}
\end{table}


\end{frame}

% ---- Frame ------------
\begin{frame}
\frametitle{Results - Hybrid ARIMA / knn 01}

% latex table generated in R 3.1.0 by xtable 1.7-3 package
% Wed Jul 09 19:05:40 2014
\begin{table}[ht]
\centering
\caption[Results from a trading system using the forecast of a continous label from a hybrid ARIMA/ANN model]{Results from a trading system using the forecast of a continous label from a hybrid ARIMA/k-NN model.} 
\label{tab:chp_ts:pUD_01_arima_knn_sys}
\begin{tabular}{lcccccc}
  \toprule Mkt & LongPL & ShortPL & L Win \% & Av L PL & S Win \% & Av S PL \\ 
  \midrule Dax & 14122 & 15787 & 64 & 11 & 55 & 7 \\ 
  CAC & 11115 & 17540 & 65 & 10 & 57 & 7 \\ 
  FTSE & 18156 & 17563 & 65 & 14 & 56 & 8 \\ 
  Dow & 28106 & 18383 & 66 & 20 & 55 & 9 \\ 
  Nikkei & 21724 & 39549 & 63 & 25 & 57 & 15 \\ 
  AORD & 9607 & 8647 & 65 & 7 & 56 & 4 \\ 
   \bottomrule \end{tabular}
\end{table}


\end{frame}

%------------------------------------------------

\begin{frame}
\Huge{\centerline{The End}}
\end{frame}

%----------------------------------------------------------------------------------------

\end{document} 